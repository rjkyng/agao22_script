\chapter{Separating Hyperplanes, Lagrange Multipliers, KKT Conditions, and Convex Duality}
% \documentclass[draft,11pt]{article}
%\documentclass[11pt]{article}
%%  RASMUS  packages
\usepackage{blindtext}
\usepackage[usenames,dvipsnames,svgnames,table]{xcolor}
\usepackage{enumitem}
\usepackage{caption}
\usepackage{placeins}
\usepackage{graphicx}

\usepackage{fancybox}
\usepackage{hyperref}

\usepackage{amsmath,amsfonts,amsthm,amssymb,xcolor}
% \let\amsboldsymbol\boldsymbol % if you want to see the bad
% boldsymbol for comparison
\usepackage{bm} % fixes boldsymbol

\usepackage{nicefrac}

\usepackage{float}


% switched to algorithm2e in lecture 9
% \usepackage{algorithm} %ctan.org\pkg\algorithms
% \usepackage{algorithmic}
% \usepackage{algorithmicx}
% \usepackage[noend]{algpseudocode}

\newtheorem{theorem}{Theorem}[section]
\newtheorem{corollary}[theorem]{Corollary}
\newtheorem{lemma}[theorem]{Lemma}
\newtheorem{observation}[theorem]{Observation}
\newtheorem{proposition}[theorem]{Proposition}
\newtheorem{claim}[theorem]{Claim}
\newtheorem{fact}[theorem]{Fact}
\newtheorem{assumption}[theorem]{Assumption}
\newtheorem{warning}[theorem]{Warning}
\newtheorem{conjecture}[theorem]{Conjecture}

\newtheorem{pseudotheorem}{Pseudotheorem}[section]
\newtheorem{pseudoclaim}[theorem]{Pseudoclaim}


\theoremstyle{definition}
\newtheorem{definition}[theorem]{Definition}
\newtheorem{remark}[theorem]{Remark}

\newtheorem*{theorem*}{Theorem}
\newtheorem*{corollary*}{Corollary}
\newtheorem*{conjecture*}{Conjecture}
\newtheorem*{lemma*}{Lemma}
\newtheorem*{thm*}{Theorem}
\newtheorem*{prop*}{Proposition}
\newtheorem*{obs*}{Observation}
\newtheorem*{definition*}{Definition}
\newtheorem*{example}{Example}
\newtheorem*{remark*}{Remark}
\newtheorem*{rec*}{Recommendation}

\newenvironment{fminipage}%
  {\begin{Sbox}\begin{minipage}}%
  {\end{minipage}\end{Sbox}\fbox{\TheSbox}}

\newenvironment{algbox}[0]{\vskip 0.2in
\noindent 
\begin{fminipage}{6.3in}
}{
\end{fminipage}
\vskip 0.2in
}


\let\muchl\ll

\def\pleq{\preccurlyeq}
\def\pgeq{\succcurlyeq}
\def\pge{\succ}
\def\ple{\prec}

\def\Approx#1{\approx_{#1}}

%\def\Span#1{\textbf{Span}\left(#1  \right)}
\def\bvec#1{{\mbox{\boldmath $#1$}}}


\newcommand\congestion{\mathit{cong}}

\def\prob#1#2{\mbox{Pr}_{#1}\left[ #2 \right]}
\def\pvec#1#2{\vec{\mbox{P}}^{#1}\left[ #2 \right]}
\def\expec#1#2{{\mathbb{E}}_{#1}\left[ #2 \right]}
\def\var#1{\mbox{\bf Var}\left[ #1 \right]}

\def\defeq{\stackrel{\mathrm{def}}{=}}
\def\setof#1{\left\{#1  \right\}}
\def\sizeof#1{\left|#1  \right|}


\def\trace#1{\mathrm{Tr} \left(#1 \right)}

\def\floor#1{\left\lfloor #1 \right\rfloor}
\def\ceil#1{\left\lceil #1 \right\rceil}

\def\dim#1{\mathrm{dim} (#1)}
\def\sgn#1{\mathrm{sgn} (#1)}

\def\union{\cup}
\def\intersect{\cap}
\def\Union{\bigcup}
\def\Intersect{\bigcap}

\def\abs#1{\left|#1  \right|}

\def\norm#1{\left\| #1 \right\|}
\def\smallnorm#1{\| #1 \|}

\newcommand\grad{\boldsymbol{\nabla}}
\newcommand\D[2]{D#1[#2]}

\newcommand*\diff{\mathop{}\!\mathrm{d}}
\newcommand*\Diff[1]{\mathop{}\!\mathrm{d^#1}}

\newcommand\ip[1]{\left< #1 \right>}


\newcommand{\sym}[1]{\mathrm{sym} (#1)}



\def\calC{\mathcal{C}}
\def\calD{\mathcal{D}}
\def\calE{\mathcal{E}}
\def\calF{\mathcal{F}}
\def\calG{\mathcal{G}}
\def\calL{\mathcal{L}}
\def\calS{\mathcal{S}}
\def\calT{\mathcal{T}}
\def\calM{\mathcal{M}}

\newcommand\calDD{\boldsymbol{\calD}}

\newcommand\DDelta{\boldsymbol{\mathit{\Delta}}}
\newcommand\Ppsi{\boldsymbol{\mathit{\Psi}}}
\newcommand\PPsi{\boldsymbol{\mathit{\Psi}}}
\newcommand\ppsi{\boldsymbol{\mathit{\psi}}}
\newcommand\pphi{\boldsymbol{\mathit{\phi}}}
\newcommand\PPhi{\boldsymbol{\Phi}}
%\newcommand\Llambda{\boldsymbol{\mathit{\Lambda}}}
\newcommand\LLambda{\boldsymbol{\mathit{\Lambda}}}
\newcommand\PPi{\boldsymbol{\Pi}}

\newcommand\ppi{\boldsymbol{\pi}}
\newcommand\cchi{\boldsymbol{\chi}}
\newcommand\aalpha{\boldsymbol{\alpha}}
\newcommand\bbeta{\boldsymbol{\beta}}
\newcommand\ggamma{\boldsymbol{\gamma}}
\newcommand\ddelta{\boldsymbol{\delta}}

\newcommand\rrho{\boldsymbol{\rho}}
\newcommand\xxi{\boldsymbol{\xi}}
%\newcommand\cchi{\boldsymbol{\chi}}

\newcommand\er{R_{\text{eff}}}


\def\aa{\pmb{\mathit{a}}}
\newcommand\bb{\boldsymbol{\mathit{b}}}
\newcommand\cc{\boldsymbol{\mathit{c}}}
\newcommand\dd{\boldsymbol{\mathit{d}}}
\newcommand\ee{\boldsymbol{\mathit{e}}}
\newcommand\ff{\boldsymbol{\mathit{f}}}
\renewcommand\gg{\boldsymbol{\mathit{g}}}
\newcommand\ii{\boldsymbol{\mathit{i}}}
\newcommand\jj{\boldsymbol{\mathit{j}}}
\newcommand\kk{\boldsymbol{\mathit{k}}}
\renewcommand\ll{\boldsymbol{\mathit{l}}}
\newcommand\pp{\boldsymbol{\mathit{p}}}
\newcommand\qq{\boldsymbol{\mathit{q}}}
\newcommand\bs{\boldsymbol{\mathit{s}}}
\newcommand\nn{\boldsymbol{\mathit{n}}}
\newcommand\rr{\boldsymbol{\mathit{r}}}
\renewcommand\ss{\boldsymbol{\mathit{s}}}
\def\tt{\boldsymbol{\mathit{t}}}
\newcommand\uu{\boldsymbol{\mathit{u}}}
\newcommand\vv{\boldsymbol{\mathit{v}}}
\newcommand\ww{\boldsymbol{\mathit{w}}}
\newcommand\yy{\boldsymbol{\mathit{y}}}
\newcommand\zz{\boldsymbol{\mathit{z}}}
\newcommand\xx{\boldsymbol{\mathit{x}}}

\newcommand\veczero{\boldsymbol{0}}
\newcommand\vecone{\boldsymbol{1}}

\newcommand\matzero{\boldsymbol{0}}
\newcommand\matone{\boldsymbol{1}}

\newcommand{\matlow}{\boldsymbol{\mathit{{\mathcal{L}}}}}
\newcommand{\matlowtil}{\boldsymbol{\mathit{\widetilde{\mathcal{L}}}}}
\newcommand{\matlowhat}{\boldsymbol{\mathit{\widehat{\mathcal{L}}}}}

\newcommand{\matup}{\boldsymbol{\mathit{{\mathcal{U}}}}}


\renewcommand\AA{\boldsymbol{\mathit{A}}}
\newcommand\BB{\boldsymbol{\mathit{B}}}
\newcommand\CC{\boldsymbol{\mathit{C}}}
\newcommand\DD{\boldsymbol{\mathit{D}}}
\newcommand\EE{\boldsymbol{\mathit{E}}}
\newcommand\GG{\boldsymbol{\mathit{G}}}
\newcommand\HH{\boldsymbol{{H}}}
\newcommand\II{\boldsymbol{\mathit{I}}}
\newcommand\JJ{\boldsymbol{\mathit{J}}}
\newcommand\KK{\boldsymbol{\mathit{K}}}
\newcommand\NN{\boldsymbol{\mathit{N}}}
\newcommand\MM{\boldsymbol{\mathit{M}}}
\newcommand\LL{\boldsymbol{\mathit{L}}}
\newcommand\PP{\boldsymbol{\mathit{P}}}
\newcommand\RR{\boldsymbol{\mathit{R}}}
\renewcommand\SS{\boldsymbol{\mathit{S}}}
\newcommand\TT{\boldsymbol{\mathit{T}}}
\newcommand\UU{\boldsymbol{\mathit{U}}}
\newcommand\WW{\boldsymbol{\mathit{W}}}
\newcommand\VV{\boldsymbol{\mathit{V}}}
\newcommand\XX{\boldsymbol{\mathit{X}}}
\newcommand\YY{\boldsymbol{\mathit{Y}}}



\newcommand\MMtil{\boldsymbol{\mathit{\tilde{M}}}}
\newcommand\AAtil{\boldsymbol{\mathit{\tilde{A}}}}
\newcommand\BBtil{\boldsymbol{\mathit{\tilde{B}}}}
\newcommand\LLtil{\boldsymbol{\mathit{\tilde{L}}}}
\newcommand\MMtilde{\boldsymbol{\mathit{\tilde{M}}}}
\newcommand\XXtil{\boldsymbol{\mathit{\tilde{X}}}}

\newcommand\AAn{\boldsymbol{\mathcal{A}}}
\newcommand\ZZ{\boldsymbol{\mathit{Z}}}

\newcommand\AAhat{\boldsymbol{\widehat{\mathit{A}}}}
\newcommand\AAapprox{\boldsymbol{\widetilde{\mathit{A}}}}
\newcommand\DDhat{\boldsymbol{\widehat{\mathit{D}}}}
\newcommand\DDapprox{\boldsymbol{\widetilde{\mathit{D}}}}
\newcommand\LLhat{\boldsymbol{\widehat{\mathit{L}}}}
\newcommand\LLapprox{\boldsymbol{\widetilde{\mathit{L}}}}
\newcommand\MMhat{\boldsymbol{\widehat{\mathit{M}}}}
\newcommand\MMapprox{\boldsymbol{\widetilde{\mathit{M}}}}
\newcommand\ZZhat{\boldsymbol{\widehat{\mathit{Z}}}}

\newcommand\DDtil{\boldsymbol{\widetilde{\mathit{D}}}}

\newcommand\fftil{\boldsymbol{\tilde{\mathit{f}}}}
\newcommand\sstil{\boldsymbol{\tilde{\mathit{s}}}}
\newcommand\xxtil{\boldsymbol{\tilde{\mathit{x}}}}
\newcommand\yytil{\boldsymbol{\tilde{\mathit{y}}}}
\newcommand\wwtil{\boldsymbol{\tilde{\mathit{w}}}}

\newcommand\ddeltatil{\boldsymbol{\tilde{\mathit{\delta}}}}


\newcommand\Otil{\widetilde{O}}

\newcommand\xhat{{\hat{{x}}}}
\newcommand\uhat{{\hat{{u}}}}
\newcommand\uuhat{\boldsymbol{\mathit{\hat{u}}}}
\newcommand\vhat{{\hat{{v}}}}
\newcommand\what{{\hat{{w}}}}

\newcommand\Ghat{{\widehat{{G}}}}
\newcommand\GGhat{\boldsymbol{\widehat{G}}}

\newcommand\Ehat{{\widehat{{E}}}}


\newcommand\R{\mathbb{R}}
\newcommand\N{\mathbb{N}}

\newcommand\ffhat{\boldsymbol{\hat{\mathit{f}}}}

\newcommand\cchat{\boldsymbol{\widehat{\mathit{c}}}}
\newcommand\sshat{\boldsymbol{\mathit{\widehat{s}}}}
\newcommand\xxhat{\boldsymbol{\mathit{\widehat{x}}}}
\newcommand\yyhat{\boldsymbol{\widehat{\mathit{y}}}}
\newcommand\xxbar{\overline{\boldsymbol{\mathit{x}}}}
\newcommand\yybar{\overline{\boldsymbol{\mathit{y}}}}
\newcommand\xxstar{{\boldsymbol{\mathit{x}}^{*}}}
\newcommand\yystar{{\boldsymbol{\mathit{y}}^{*}}}


\newcommand\ffbar{\overline{\boldsymbol{\mathit{f}}}}


\newcommand\ddeltabar{\boldsymbol{\bar{\mathit{\delta}}}}
\newcommand\ddeltahat{\boldsymbol{\hat{\mathit{\delta}}}}



\newcommand\energy{\mathcal{E}}


\newcommand{\todo}[1]{{\bf \color{red} TODO: #1}}
\newcommand{\todolow}[1]{{\bf \color{orange} TODOLOW: #1}}

\newcommand{\richard}[1]{{\bf \color{green} Richard: #1}}
\newcommand{\rasmus}[1]{{\bf \color{olive} Rasmus: #1}}
\newcommand{\ahad}[1]{{\bf \color{olive} Ahad: #1}}


\newcommand{\expct}[2]{{}\mathop{\mathbb{E}}_{#1}\left[#2\right]}
\newcommand{\E}[1]{\mathop{{}\mathbb{E}}\left[#1\right]}
%% https://tex.stackexchange.com/questions/56765/getting-the-expectation-symbol-to-behave-like-sum-instead-of-sigma
%% extra {} inside mathop gives nicer (standard) vertical placement of E
\newcommand{\Var}[1]{\mathop{{}Var}\left[#1\right]}
\newcommand{\Ex}[1]{{}\mathop{\mathbb{E}}_{#1}}
\newcommand\tr{\mathrm{Tr}}


%%%%% LINEAR ALGEBRA

\newcommand{\schurto}[2]{\ensuremath{\textsc{Sc}\!\left[#1\right]_{#2}}}
% \newcommand{\schurto}[2]{\ensuremath{\textsc{Sc}\left[#1\right]_{#2}}}

%{$\textsc{Sc}\left[#1\right]_{#2}$}
% \newcommand{\schurto}[2]{\textsc{Sc}\left(#1, #2\right)}
%\newcommand{\schurto}[2]{\left[{#1}\right]_{\to #2}}
\renewcommand{\sc}[2]{\schurto{#1}{#2}}

%transpose
\newcommand{\trp}{\top}

%pseudoinverse
\newcommand{\pinv}{+}


\newcommand{\proj}{\PPi}

\DeclareMathOperator{\nnz}{nnz}

\DeclareMathOperator*{\argmin}{arg\,min}
\DeclareMathOperator*{\argmax}{arg\,max}
\DeclareMathOperator*{\val}{val}

\DeclareMathOperator*{\diag}{diag}
\DeclareMathOperator*{\Span}{span}

%\DeclareMathOperator*{\ker}{ker}
\DeclareMathOperator*{\im}{im}

%%%%% LAYOUT

\newenvironment{tight_enumerate}{
\begin{enumerate}
 \setlength{\itemsep}{2pt}
 \setlength{\parskip}{1pt}
}{\end{enumerate}}
\newenvironment{tight_itemize}{
\begin{itemize}
 \setlength{\itemsep}{2pt}
 \setlength{\parskip}{1pt}
}{\end{itemize}}
\newenvironment{tight_description}{
\begin{description}
 \setlength{\itemsep}{2pt}
 \setlength{\parskip}{1pt}
}{\end{description}}

\newcommand*{\vertbar}{\rule[-1ex]{0.5pt}{2.5ex}}
\newcommand*{\horzbar}{\rule[.5ex]{2.5ex}{0.5pt}}

%Basics
\newcommand{\new}[1]{{\em #1\/}}		% New term (set in italics).

\newcommand{\boxwidth}{\dimexpr\linewidth-2em\relax}

\newcommand{\boxdef}[1]
{
\fbox{
\begin{minipage}{\boxwidth}
\begin{definition}
{#1}
\end{definition}
\end{minipage}
}
}

\newcommand{\boxthm}[1]
{
\fbox{
\begin{minipage}{\boxwidth}
\begin{theorem}
{#1}
\end{theorem}
\end{minipage}
}
}

\newcommand{\boxfact}[1]
{
\fbox{
\begin{minipage}{\boxwidth}
%\begin{theorem*}
\emph{{#1}
%\end{theorem*}
}
\end{minipage}
}
}

%\newcommand{\handout}[5]{
  \noindent
  \begin{center}
  \framebox{
    \vbox{
            \hbox to 6.30in { {\bf Advanced Graph Algorithms and Optimization} \hfill #2 }
            \vspace{5mm}
            \hbox to 6.30in { {\Large \hfill #5  \hfill} }
            \vspace{3mm}
            \hbox to 6.30in { {\em #3 \hfill #4} }
    }
  }
  \end{center}
  \vspace*{4mm}
}

\newcommand{\lecture}[4]{\handout{#1}{#2}{#3}{Lecture #1}{#4}}
\newcommand{\homework}[3]{\handout{#1}{#2}{#3}{Problem Set #1}}
\newcommand{\gradedhomework}[3]{\handout{#1}{#2}{#3}{Graded Homework #1}}
\newcommand{\sect}[3]{\handout{#1}{#2}{#3}{Section #1}}




% 1-inch margins, from fullpage.sty by H.Partl, Version 2, Dec. 15, 1988.
\topmargin 0pt
\advance \topmargin by -\headheight
\advance \topmargin by -\headsep
\textheight 8.9in
\oddsidemargin 0pt
\evensidemargin \oddsidemargin
\marginparwidth 0.5in
\textwidth 6.5in

\parindent 0in
\parskip 1.5ex
 
%\usepackage{amsmath}
%\usepackage{amssymb}
%\usepackage{amsthm}
%\usepackage{graphicx}
%\usepackage{float}
%\usepackage[ruled,vlined]{algorithm2e}
%\SetKwBlock{Repeat}{repeat}{}
%%% for this lecture
%\newcommand{\gap}{\text{gap}}
% \newtheorem{theorem}{Definition}
%\interfootnotelinepenalty=10000
% \newtheorem{lemma}{Lemma}

%\usepackage{tikz}
%\usetikzlibrary{arrows}


%%%% ADD MACROS HERE
% feel free to add more macros here

%\begin{document}

\sloppy
%\lecture{12 --- Wednesday, May 13th}
%{Spring 2020}{Rasmus Kyng, Scribe: Tim Taubner}{
% Separating Hyperplanes, Lagrange Multipliers, and Convex Duality}

\section{Overview}
First part of this chapter introduces the concept of a separating hyperplane of two sets followed by a proof that for two closed, convex and disjoint sets a separating hyperplane always exists.
This is a variant of the more general \emph{separating hyperplane theorem}%
\footnote{Wikipedia is good on this: \href{https://en.wikipedia.org/wiki/Hyperplane_separation_theorem}{https://en.wikipedia.org/wiki/Hyperplane\_separation\_theorem}} %
due to Minkowski.
Then \emph{Lagrange multipliers} $\xx$, $\ss$ of a convex optimization
problem
\begin{align*}
    \min_y{}\ & \energy(\yy) \\ \nonumber
\text{s.t.}\  & \AA\yy = \bb \\ \nonumber
              & c(\yy) \leq 0
\end{align*}
are introduced and with that, the \emph{Lagrangian} \begin{equation*} \LL(\yy, \xx, \ss) = \energy(\yy) + \xx^{\trp}(\bb - \AA\yy) + \ss^{\trp}c(\yy) \end{equation*}
is defined.
Finally, we deal with the dual problem
\begin{align*}
     \max_{\xx, \ss, \ss \geq 0}{} L(\xx,  \ss),
\end{align*}
where $L(\xx, \ss) = \min_y L(\yy, \xx, \ss)$. We show \emph{weak
  duality}, i.e.\ $L(\yy, \xx, \ss) \leq \energy(\yy)$ and that assuming \emph{Slater's condition} the values of both the primal and dual is equal, which is referred to as \emph{strong duality}.

\section{Separating Hyperplane Theorem}
Suppose we have two convex subsets $A, B \subseteq \R^n$ that are disjoint ($A \cup B = \emptyset$).
We wish to show that there will always be a (hyper-)plane $H$ that separates these two sets, i.e.~ $A$ lies on one side, and $B$ on the other side of $H$.

So what exactly do we mean by Hyperplane? Let's define it.
\begin{definition}[Hyperplane]
A \emph{hyperplane} $H$ of dimension $n$ is the subset $H := \{\xx \in \R^n : \ip{\nn, \xx} = \mu\}$.
We say $H$ has \emph{normal} $\nn \in \R^n$ and \emph{threshold} $\mu$.
It is required that $\nn \neq \veczero$.
\end{definition}

Every hyperplane divides $\R^n$ into two halfspaces $\{\xx : \ip{\vv, \xx} \geq \mu\}$ and $\{\xx : \ip{\vv, \xx} \leq \mu\}.$
It separates two sets, if they lie in different halfspaces. We formally define separating hyperplane as follows.

\begin{definition}[Separating Hyperplane]
We say a hyperplane $H$ \emph{separates} two sets $A, B$ iff
\begin{align*}
  \forall \aa \in A:& \ip{\nn, \aa} \geq \mu \\
  \forall \bb \in B:& \ip{\nn, \bb} \leq \mu
\end{align*}
If we replace $\geq$ with $>$ and $\leq$ with $<$ we say $H$ \emph{strictly} separates $A$ and $B$.
\end{definition}

It is easy to see that there exists disjoint non-convex sets that can not be
separated by a hyperplane (e.g.\ a point cannot be separated from a
ring around it). But can two disjoint convex sets always be strictly separated by a hyperplane?
The answer is no: consider the two-dimensional case depicted in
\autoref{fig:lefthyper} with $A = \{ (x,y): x \leq 0\}$ and $B = \{
(x, y) : x> 0 \text{ and } y \geq \frac{1}{x}\}$. Clearly they are disjoint; however the only separating hyperplane is $H = \{ (x, y): x = 0\}$ but it intersects $A$.

\begin{figure}[ht]
  \centering
  \includegraphics[width=1\textwidth]{fig/lec12-non-strict-separator.png}
  \caption{The sets $A = \{ (x, y): x \leq 0\}$ and $B = \{ (x, y): x
    > 0 \text{ and } y \geq \frac{1}{x} \}$ only permit a non-strictly separating hyperplane.}
  \label{fig:lefthyper}
\end{figure}

One can prove that there exists a non-strictly separating hyperplane for any two disjoint convex sets.
We will prove that if we further require $A$,$B$ to be closed and
bounded, then a strictly separating hyperplane always exists. (Note in
the example above how our choice of $B$ is not bounded.)

\begin{theorem}[Separating Hyperplane Theorem; closed, bounded sets] \label{th:shtcb}
For two closed, bounded, and disjoint convex sets $A, B \in \R^n$ there exists a strictly separating hyperplane $H$.
One such hyperplane is given by normal $\nn = \dd - \cc$ and threshold $\mu = \frac{1}{2}\left(\norm{\dd}_2^2 - \norm{\cc}_2^2\right)$, where $\cc \in A$, $\dd \in B$ are the minimizers of the distance between $A$ and $B$
\begin{equation*} \text{dist}(A, B) = \min_{\aa\in A, \bb \in B}\norm{\aa-\bb}_2 > 0.\end{equation*}
% Note that this distance is strictly larger than zero.
\end{theorem}
\begin{proof}
  We omit the proof that $\text{dist}(A, B) = \min_{\aa\in A, \bb \in
    B}\norm{\aa-\bb}_2 > 0$, which follows from $A,B$ being disjoint,
  closed, and bounded.
Now, we want to show that $\ip{\nn,\bb} > \mu$ for all $\bb \in B$; then $\ip{\nn,\aa} < \mu$ for all $\aa\in A$ follows by symmetry.
Observe that
\vspace{-1em}
\begin{align*}
  \ip{\nn, \dd} - \mu&= \ip{\dd-\cc,\dd} - \frac{1}{2}\left(\norm{\dd}_2^2-\norm{\cc}_2^2\right)\\
                    &= \norm{\dd}_2^2 - \dd^{\trp}\cc - \frac{1}{2}\norm{\dd}_2^2 + \frac{1}{2}\norm{\cc}_2^2 \\
										&= \frac{1}{2}\norm{\dd-\cc}_2^2 > 0.
\end{align*}
So suppose there exists $\uu \in B$ such that $\ip{\nn, \uu} - \mu \leq 0$.
We now look at the line defined by the distance minimizer $\dd$ and the point on the ``wrong side'' $\uu$.
Define $\bb(\lambda) = \dd + \lambda(\uu - \dd)$, and take the derivative of the distance between $\bb(\lambda)$ and $\cc$.
Evaluated at $\lambda=0$ (which is when $\bb(\lambda) = \dd$), this yields
\begin{equation*}
  \left.\frac{d}{d\lambda} \norm{\bb(\lambda) - \cc}_2^2\right\rvert_{\lambda=0} =
	\left.2\ip{\dd-\lambda\dd+\lambda\uu-\cc,\uu-\dd}\right\rvert_{\lambda=0} =
	2\ip{\dd-\cc, \uu-\dd}.
\end{equation*}

However, this would imply that the gradient is strictly negative since
\begin{align*}
\ip{\nn, \uu} - \mu & = \ip{\dd-\cc,\uu} - \ip{\dd-\cc,\dd} + \ip{\dd-\cc,\dd} - \mu \\
                  & = \ip{\dd-\cc,\uu-\dd} + \norm{\dd}_2^2 - \ip{\cc,\dd} - \frac{1}{2}\norm{\dd}_2^2+\frac{1}{2}\norm{\cc}_2^2 \\
                  & = \ip{\dd-\cc,\uu-\dd} + \frac{1}{2}\norm{\dd-\cc}_2^2 \leq 0.
\end{align*}
This contradicts the minimality of $\dd$ and thus concludes this proof.
\end{proof}

A more general separating hyperplane theorem holds even when the sets
are not closed and bounded:

\begin{theorem}[Separating Hyperplane Theorem] \label{th:sht}
Given two disjoint convex sets $A, B \in \R^n$ there exists a hyperplane $H$
separating them.
\end{theorem}


\section{Lagrange Multipliers and Duality of Convex Problems}

In this Section, we'll learn about \emph{Langrange Multipliers} and
how they lead to convex duality.
But first, let's see an example to help illustrate where these ideas
come from.

Imagine you were to prove that for all $\xx \in \R^n$ we have $\norm{\xx}_p \leq n^{\frac{1}{2} - \frac{1}{p}}\norm{\xx}_2$ for some $1 \leq p \leq 2;$.
We can look at this as optimizing $\max_{\xx} \norm{\xx}_p$ subject to $\norm{\xx}_2$ being constant, e.g.\ simply $\norm{\xx}_2 = 1$. Then the statement above follows from a scaling argument.

\begin{figure}[ht]
  \centering
  \includegraphics[width=.48\textwidth]{fig/lec12-2p-norms.jpeg}
  \caption{Looking at fixed $\norm{\xx}_p = \alpha$ and $\norm{\xx}_2=1$. (Here, $p = 1.5$.)}
  \label{fig:2-p-norm}
\end{figure}

If we move from $\xx$ to $\xx + \ddelta$ with $\ddelta \perp \grad_{\xx}\norm{\xx}_2$ and $\ddelta \not\perp \grad_{\xx}\norm{\xx}_p$ means that for infinitesimally small $\ddelta$ the 2-norm stays constant but the $p$-norm changes.
That means for either $\xx - \ddelta$ or $\xx + \ddelta$ the $p$-norm increases
while the 2-norm stays constant. Hence at the maximum of $\norm{\xx}_p$ the gradients of both norms have to be parallel, i.e.~\begin{equation*}\grad_{\xx}\left(\norm{\xx}_p - \lambda\norm{\xx}_2\right) = 0.\end{equation*}
%Here, $\lambda$ is the
This insight is the core idea of Lagrange multipliers (in this case $\lambda$).

Note that here the problem is not convex, because $\{ \xx: \norm{\xx}_2^2 =
1 \}$ is not convex and because we are asking to \emph{maximize} a norm.
In the following we will study Lagrange multipliers for general convex
problems.

\subsection{Karush-Kuhn Tucker Optimality Conditions for Convex Problems}

A full formal treament of convex duality would require us to be more careful
about using $\inf$ and $\sup$ in place of $\min$ and $\max$, as well
as considering problems that have no feasible solutions.
Today, we'll ignore these concerns.

Let us consider a general convex optimization problem with convex objective, linear equality constraints and convex inequality constraints
\begin{align}
  \label{eq:primalprob}
     \min_{y \in S}{} & \energy(\yy) \\ \nonumber
\text{s.t.}\  & \AA\yy = \bb \\ \nonumber
              & \cc(\yy) \leq \veczero,
\end{align}
where $\energy(\yy): S \rightarrow \R$ is defined on a convex subset $S \subseteq \R^n$, $\AA \in \R^{m\times n}$
and $\cc(\yy)$ is a vector of constraints $\cc(\yy) = \left(c_i(\yy)\right)_{i \in [k]}$.
For every $i \in [k]$ the function $c_i: S \rightarrow \R$ should be convex,
which implies the sublevel set $\setof{y : c_i(\yy) \leq 0 }$ is
convex.
Given a solution $\yy$, we say an inequality constraint is
\emph{tight} at $\yy$ if $c_i(\yy) = 0$.
In the following we will denote by $\alpha^* = \energy(\yy^*)$ the optimal value of this program where $\yy^*$ is a minimizer.

We will call this \emph{the primal program} -- and later we will see
that we can associate another related convex program with any such
convex program -- and we will call this second program \emph{the dual program}.

\begin{definition}[Primal feasibility]
We say that $\yy \in S$ is \emph{primal feasible} if all constraints are satisfied, i.e.~$\AA\yy = \bb$ and $\cc(\yy) \leq \veczero$.
\end{definition}

Now, as we did in our example with the 2- and $p$-norms, we will try
to understand the relationship between the gradient of the objective
function and of the constraint functions at an optimal solution $\yy^*$.

\paragraph{An (not quite true!) intuition.}
Suppose $\yy^*$ is an optimal solution to the convex program above.
Let us additionally suppose that $\yy^*$ is \emph{not} on the boundary
of $S$.
Then, generally speaking, because we are at a constrained minimum of
$\energy(\yy^*)$, we must have that for any infinitesimal $\ddelta$
s.t. $\yy^*+\ddelta$ is also feasible, $\ddelta^{\top} \grad
\energy(\yy^*) \geq 0$, i.e. the infinitesimal does not decrease the
objective.
We can also view this as saying that if $\ddelta^{\top} \grad
\energy(\yy^*)  < 0$, the update must be infeasible.
But what kind of updates will make $\yy^*+\ddelta$ infeasible?
This will be true if $\aa_j^{\top} \ddelta \neq 0$ for some linear
constraint $j$ or, roughly speaking, $\grad c_i(\yy^*)^{\top} \ddelta \neq 0$ for some
\emph{tight} inequality constraint.
But, if this is true for \emph{all} directions that have a negative
inner product with $\grad \energy(\yy^*)$, then we must have that
$-\grad \energy(\yy^*)$ can be written as a linear combination of
$\aa_j$, i.e. gradients of the linear constraint, and of
gradients $\grad c_i(\yy^*)$ of tight inequality constraints,
and furthermore the cofficients of the gradients of these tight
inequality constraints must be \emph{positive}, so that moving along
this direction will increase the function value and violate the constraint. 

To recap, given cofficients $\xx \in \R^m$ and $\ss \in \R^k$ with
$\ss(i) \geq 0$ if $c_i(\yy^*) = 0$, and $\ss(i) = 0$ otherwise, we
should be able to write
\begin{align}
  -\grad_{\yy}\calE(\yy^*) = \sum_{j} \xx(j) \aa_j + \sum_{i}
  \ss(i)\grad c_i(\yy^*)
  \label{eq:compSlackIntro}
\end{align}
  Note that since $\yy^*$ is feasible, and hence $\cc(\yy^*) \leq 0$,
  we can write the condition that
$\ss(i) \geq 0$ if $c_i(\yy^*) = 0$, and $\ss(i) = 0$ otherwise, in a
very slick way: namely as $\ss \geq \veczero$ and $\ss^{\top}
\cc(\yy^*) = 0$.
Traditionally, the variables in $\ss$ are called \emph{slack
  variables}, because of this, i.e. they are non-zero only if there is
no \emph{slack} in the constraint. 
This condition has a fancy name: when $\ss^{\top}
\cc(\yy) = 0$ for some feasible $\yy$ and $\ss \geq \veczero$, we say
that $\yy$ and $\ss$ satisfy \emph{complementary slackness}.
%
We will think of the vectors $\ss$ and $\xx$ as variables that help
us prove optimality of a current solution $\yy$, and we call them
\emph{dual variables}.

\begin{definition}[Dual feasibility]
We say $(\xx, \ss)$ is dual feasible if $\ss \geq 0$.
If additionally $\yy$ is primal feasible, we say $(\yy, \xx, \ss)$ is primal-dual feasible.
\end{definition}

Now, we have essentially argued that at any optimal solution
$\yy^*$, we must have that complementary slackness holds, and that
Equation~\eqref{eq:compSlackIntro} holds for some $\xx$ and some $\ss
\geq \veczero$.
However, while this intuitive explanation is largely correct, it turns
out that it can fail for technical reasons in some weird
situations\footnote{
  Consider the following single-variable optimization problem
  \begin{align*}
    \min_{x
    \in \R} &\, x
    \\
    \text{s.t.} & \, x^2 = 0.
  \end{align*}
  This has only a single feasible point $x = 0$, which must then be
  optimal.
  But at this point, the gradient of the constraint function is zero,
  while the gradient of the objective is non-zero.
  Thus our informal reasoning breaks down, because there exists an infeasible
  direction $\delta$ we can move along where the constraint function grows, but
  at a rate of $O(\delta^2)$.
  }.
Nonetheless, under some mild conditions, it is indeed true that the 
conditions we argued for above must hold at any optimal solution.
These conditions have a name: The Karush-Kuhn-Tucker Conditions.
For convenience, we will state the conditions using $\grad \cc(\yy)$
to denote the matrix whose $i$th column is given by $\grad c_i(\yy)$.
This is sometimes called the Jacobian of $\cc$.

\begin{definition}[The Karush-Kuhn-Tucker (KKT) Conditions]
  Given a convex optimization problem of form
  \eqref{eq:primalprob} where the domain $S$ is \emph{open}.
  Suppose $\yy, \xx, \ss$ satisfy the following conditions:
  \begin{itemize}
  \item $\AA\yy = \bb$ and $\cc(\yy) \leq 0$ \hfill (primal feasibility)
  \item $\ss \geq 0$ \hfill (dual feasibility)
  \item $\grad_{\yy}\calE(\yy) + \AA^\trp\xx +
    \grad\cc(\yy) \ss = \matzero$ \hfill
    (KKT gradient condition, i.e. Eq.~\eqref{eq:compSlackIntro} restated)
    % $\left(\grad_{\yy}L(\yy,\xx, \ss)=\matzero\right)$
  \item $\ss(i)\cdot\cc_i(\yy) = 0$ for all $i$ \hfill (complementary slackness)
  \end{itemize}
  Then we say that $\yy, \xx, \ss$ satisfy the  \emph{Karush-Kuhn-Tucker} (KKT) conditions.
\end{definition}

\subsection{Slater's Condition}
There exists many different mild technical conditions under which the
KKT conditions do indeed hold at any optimal solution $\yy$.
The simplest and most useful is probably \emph{Slater's condition}.

\begin{definition}[Slater's condition with full domain] \label{def:slater}
A (primal) problem as defined in \eqref{eq:primalprob} with $S = \R^n$ fulfills
Slater's condition if there exists a \emph{strictly feasible} point,
i.e.\ there exists $\yytil$ s.t.\ $\AA\yytil = \bb$ and $\cc(\yytil) <
\veczero$.
This means that the strictly feasible point $\yytil$ lies strictly inside the set $\{\yy : \cc(\yy) \leq \veczero\}$ defined by the inequality constraints.
\end{definition}

One way to think about Slater's condition is that your inequality
constraints should not restrict the solution space to be
lower-dimensional. This is a degenerate case as the sublevel sets of
the inequality constraints are generically full-dimenensional and you
want to avoid this degenerate case.

We can also extend Slater's condition to the case when the domain $S$
is an open set.
To extend Slater's condition to this case, we need the notion of a ``relative interior''.

% \begin{definition}[Affine hull] \label{def:affhull}
% Given a set $S \subset \R^n$, the \emph{affine hull} of $S$, denoted $\operatorname{aff}(S)$, is the
% union of $S$ and all lines through any pair of points in $S$.
% We can also write this as
% \[
%   \operatorname{aff}(S) = \setof{\alpha \xx + (1-\alpha) \yy : \xx,
%     \yy \in S, \alpha \in \R}. 
%   \]
% \end{definition}

\begin{definition}[Relative interior] \label{def:relint}
  Given a convex set $S \subset \R^n$, the \emph{relative interior} of $S$ is
  \[
    \operatorname{relint}(S) = \setof{x \in S : \text{ for all } \yy
      \in S \text{ there exists } \epsilon > 0 \text{ such that }
      \xx - \epsilon(\yy - \xx) \in S}
    .
  \]
\end{definition}
In other words, $\xx \in \operatorname{relint}(S)$ if starting at $\xx
\in S$ we can move
``away'' from any $\yy \in S$ by a little and still be in $\SS$.
As an example, suppose $S = \setof{(s,t) \in \R^2 \text{ such that } s
  \geq 0 \text{ and } t = 0}$.
Then $(0,0) \in S$ but $(0,0) \not\in \operatorname{relint}(S)$, while
$(1,0) \in \operatorname{relint}(S)$.

Now, we can state a more general version of Slater's condition.

\begin{definition}[Slater's condition] \label{def:slatergeneral}
A (primal) problem as defined in \eqref{eq:primalprob} fulfills
Slater's condition if there exists a \emph{strictly feasible} point
$\yytil \in \operatorname{relint}(S)$.
We require $\AA\yytil = \bb$ and $\cc(\yytil) <
\veczero$.
This means that the strictly feasible point $\yytil$ lies strictly inside the set $\{\yy : \cc(\yy) \leq \veczero\}$ defined by the inequality constraints.
\end{definition}

Finally, we end with a proposition that tells us that given Slater's
condition, the KKT are indeed necessary for optimality of our convex programs.

\begin{proposition}[KKT necessary for optimality when Slater's condition holds]
  \label{pro:KKTnecessary}
  Consider a convex program in the form \eqref{eq:primalprob} that
  satisfies Slater's condition and has an open set $S$ as its domain.
  Suppose $\yy$ is a primal optimal (feasible) solution, and that
  $\yy, \xx, \ss$ satisfy the KKT conditions.
\end{proposition}

We will prove this lemma later, after developing some tools we will
use in the proof.
In fact, we will also see later that assuming Slater's condition, the KKT
conditions are sufficient for optimality.

\subsection{The Lagrangian and The Dual Program}

Notice that we can also rewrite the KKT gradient condition as
\[
  \grad_{\yy}
  \left[ \underbrace{\energy(\yy) + \xx^{\trp}(\bb-\AA\yy) +
      \ss^{\trp}\cc(\yy)
    }_{(\star)}
    \right]
= \veczero
  \]
That is, we can write this condition as the gradient of the quantity
$(\star)$ is zero.
But what is this quantity $(\star)$?
We call it \emph{the Lagrangian} of the program.

\begin{definition}
Given a convex program~\eqref{eq:primalprob}, we define the
\emph{Lagrangian} of the program as
\begin{equation*} L(\yy, \xx, \ss) = \energy(\yy) +
  \xx^{\trp}(\bb-\AA\yy) + \ss^{\trp}\cc(\yy). \end{equation*}

We also define a Lagrangian only in terms of the dual variables by minimizing over $\yy$ as
\begin{equation*} L(\xx, \ss) = \min_{\yy}L(\yy, \xx,
  \ss). \end{equation*}

\end{definition}
When $\ss \geq 0$, we have that $L(\yy,\xx,\ss)$ is a
convex function of $\yy$.
For each $\yy$, the Lagrangian $L(\yy, \xx,
\ss)$ is linear in $(\xx,\ss)$ and hence also concave in them.
Hence $L(\xx,  \ss)$ is a concave function, because it is the pointwise
minimum (over $\yy$), of a collection of concave functions in
$(\xx,\ss)$.
% Notice also that when $L(\yy,\xx,\ss)$ is affine in $\xx,\ss$ and hence $L(\xx, \ss) = \min_{\yy}L(\yy, \xx,
%   \ss)$ is concave -- because a minimum over concave functions is
%   concave, in the same way a maximum over convex functions is convex.

We can think of $\xx$ as assigning a price to violating the linear
constraints, and of $\ss$ as assigning a price to violating the
inequality constraints.
The KKT gradient condition tells us that at the given prices, the
there is no benefit gained from locally violating the constraints --
i.e. changing the primal solution $\yy$ would not improve the cost.

Notice that if $\yy,\xx,\ss$ are primal-dual feasible, then
\begin{align}
  \energy(\yy)
  &= \energy(\yy) + \xx^{\trp}(\bb-\AA\yy)  \tag*{as
  $\bb-\AA\yy = \veczero$.}\\
    &\geq \energy(\yy) + \xx^{\trp}(\bb-\AA\yy) + \ss^{\top} \cc(\yy)  \tag*{as
      $\cc(\yy) \leq \veczero$ and $\ss \geq \veczero$.}\\
  &= L(\yy,\xx,\ss) \label{eq:weakdual}
\end{align}

Thus, for primal-dual feasible variables, the Lagrangian is always a
lower bound on the objective value.


$L(\xx, \ss)$ is defined by minizming $L(\yy,\xx,\ss)$ over $\yy$,
i.e. what is the worst case value of the lower bound $L(\yy,\xx,\ss)$
across all $\yy$.
We can think of this as computing how good the given ``prices'' are at
approximately enforcing the constraints.
This naturally leads to a new optimization problem: How can we choose
our prices $\xx, \ss$ to get the best (highest) possible lower bound?

\begin{definition}[Dual problem]
We define the \emph{dual problem} as
\begin{align}
  \label{eq:dualprob}
  \max_{\substack{\xx, \ss \\ \ss \geq 0}}
  \min_{\yy}L(\yy, \xx, \ss)
  =
  \max_{\substack{\xx, \ss \\ \ss \geq 0}} L(\xx,  \ss)
  \end{align}
and denote the optimal dual value by $\beta^*$.
\end{definition}

The dual problem is really a convex optimization
problem in disguise, because we can flip the sign of $-L(\xx,  \ss)$
to get a convex function and minimizing this is equivalent to
maximizing $L(\xx,  \ss)$.
\begin{align*}
  \max_{\substack{\xx, \ss \\ \ss \geq 0}} L(\xx,  \ss)
  =
  -
  \min_{\substack{\xx, \ss \\ \ss \geq 0}} -L(\xx,  \ss)
\end{align*}
When $\xx$ and $\ss$ are optimal for the dual program, we
say they are dual optimal. And for convenience, when we also have a
primal optimal $\yy$, altogether, we will say that $(\yy,\xx,\ss)$ are
primal-dual optimal.

% Of course, we can write the dual problem using the
% $\min_{\yy}L(\yy, \xx, \ss)$ notation as
% \[
%   \max_{\substack{\xx, \ss \\ \ss \geq 0}}
%   \min_{\yy}L(\yy, \xx, \ss)
%   =
%   \max_{\substack{\xx, \ss \\ \ss \geq 0}} L(\xx,  \ss)
%   \]

The primal problem can also be written in terms of the Lagrangian.
\begin{equation}
  \label{eq:primalminmax}
  \alpha^* = \min_{\yy}\max_{\xx; \ss\geq\veczero} L(\yy, \xx, \ss)
\end{equation}
This is because for a minimizing $\yy$ all constraints have to be satisfied and the Lagrangian simplifies to $L(\yy,\xx,\ss) = \energy(\yy)$.
If $\AA\xx - \bb = \veczero$ was violated, making $\xx$ large sends $L(\yy,\xx,\ss) \rightarrow \infty$.
And if $\cc(\yy) \leq \veczero$ is violated, we can make $L(\yy,\xx,\ss) \rightarrow \infty$ by choosing large $\ss$.

Note that we require $\ss \geq \veczero$, as we only want to penalize the violation of the inequality constraints in one direction, i.e.\ when $\cc(\yy) > 0$.

%\begin{theorem}[Weak Duality]
For any primal-dual feasible $\yy,\xx,\ss$ we have
$L(\yy, \xx, \ss) \leq \energy(\yy)$ (see Equation~\eqref{eq:weakdual}) and hence also $L(\xx, \ss) = \min_y L(\yy, \xx, \ss) \leq \energy(\yy)$.

In other words $\max_{\xx; \ss \geq 0} L(\xx, \ss) = \beta^* \leq \alpha^*$.
This is referred to as \emph{weak duality}.

Using the forms in Equations~\eqref{eq:dualprob} and
\eqref{eq:primalminmax}, we can also state this as
\begin{theorem}[The Weak Duality Theorem]
  For any convex program~\eqref{eq:dualprob} and its dual, we have
  \begin{equation*}
  \alpha^* = \min_{\yy}\max_{\xx; \ss\geq\veczero} L(\yy, \xx, \ss)
  \geq
  \max_{\xx; \ss\geq\veczero} \min_{\yy} L(\yy, \xx, \ss)
  =
  \beta^*
  .
\end{equation*}
\end{theorem}


\subsection{Strong Duality}
So now that we have proved weak duality $\beta^* \leq \alpha^*$,
what is strong duality? $\beta^* = \alpha^*$?
The answer is yes, but strong duality only holds under some conditions.
Again, a simple sufficient condition is Slater's condition (Definition~\ref{def:slatergeneral}.

\begin{theorem}
  \label{thm:slaterstrongduality}
For a program~\eqref{eq:primalprob} satisfying Slater's condition, strong duality holds, i.e.\ $\alpha^* = \beta^*$.
In other words, the optimal value of the primal problem $\alpha^*$ is
equal to the optimal value of the dual.
\end{theorem}

Note that at primal optimal $\yy^*$ and dual optimal $\xx^*,
\ss^*$, we have
\[
  \alpha^* = \energy(\yy^*) \geq L(\yy^*, \xx^*, \ss^*)
  \geq \beta^* = \alpha^*
  .
\]
Thus, we can conclude that $L(\yy^*, \xx^*, \ss^*) = \alpha^* = \beta^*$.



% Theorem~\ref{thm:slaterstrongduality} also holds given the general
% Slater's condition of Definition~\ref{def:slatergeneral}.

\paragraph{How are we going to prove this?} Before we prove the theorem, let's make a few observations to get us
warmed up. If you get bored, skip ahead to the proof.

It is sufficient to prove that $\alpha^* \leq \beta^*$, as the statement then follows in conjunction with weak duality.
We define the set
\begin{equation*} G = \{(\energy(\yy), \AA\yy - \bb, \cc(\yy)) : \yy \in S \}, \end{equation*}
where $S \subseteq \R^n$ is the domain of $\energy$.

Immediately, we observe that we can write the optimal primal value as
\begin{equation*} \alpha^* = \min \{t:(t,\vv,\uu) \in G, \vv = \veczero, \uu \leq \veczero\}. \end{equation*}
Similarly, we can write the Lagrangian (after minimizing over $\yy$)
\begin{equation*} L(\xx, \ss) = \min_{(t,\vv,\uu) \in G} (1, \xx, \ss)^{\trp} (t, \vv, \uu). \end{equation*}
This is equivalent to the inequality, for $(t,\vv,\uu) \in G$,
\begin{equation*} (1, \xx, \ss)^{\trp}(t, \vv, \uu) \geq L(\xx, \ss). \end{equation*}
which defines a hyperplane with $\nn = (1,\xx,\ss)$ and $\mu = L(\xx,
\ss)$ such that $G$ is on one side.

To establish strong duality, we would like to show the existence of a
hyperplane such that for $(t,\vv,\uu) \in G$
\[
  \nn^{\trp}(t, \vv, \uu) \geq \alpha^* \text{ and } \nn = (1,\xxhat,\sshat)
  \text{ with } \sshat \geq \veczero.
\]
Then we would immediately get
\begin{equation*}
 \beta^* \geq L(\xxhat,\sshat)= \min_{(t,\vv,\uu) \in G} (1, \xx,
 \ss)^{\trp} (t, \vv, \uu) \geq \alpha^* .
\end{equation*}

Perhaps not surprisingly, we will use the Separating
Hyperplane Theorem.
What are the challenges we need to deal with?
\begin{itemize}
\item We need to replace $G$ with a convex set (which we will call
  $A$) and separate $A$ from some other convex set (which we will call $B$).
\item We need to make sure the hyperplane normal $\nn$ has $1$ in the first
  coordinate and $\ss \geq \veczero$, and the hyperplane threshold is $\alpha^*$.
\end{itemize}

% The proof we are about to see handles these things in one neat
% stroke: Essentially, we replace $G$ with the convex set that comes from replacing
% the convex functions $\energy$ and $\cc$ with their epigraphs.

% If strong duality indeed holds,
% and is obtained at some $L(\yyhat,\xxhat,\sshat)$
% then the reasoning we used to prove weak duality in
% Section~\ref{sec:weakdual} seems to suggest that a dual point
% $(\xx^*,\ss^*)$ maximizing $L(\xx,\ss)$ will occur at a feasible
% $\yy$, since otherwise $L(\yy,\xx,\ss)$ could be reduced further by
% changing $(\xx,\ss)$ to penalize $\yy$.
% Of course, it's not clear whether this could then be counter-acted by
% changing $\yy$.


% \begin{itemize}
% \item and correspond to a feasible $\yy$ in the sense that
%   $L(\xx^*,\ss^*) = (\yy,\xx^*,\ss^*)$
%   \item be dual feasible,
% \end{itemize}
% If the minimizing $\yy$ is not feasible,  but



%  Then by considering a feasible $\yyhat$ for the primal problem, we
%   get
%   \begin{align**}
%     \beta^*
%     & \geq L(\xxhat,\sshat) \\
%     &= (1, \xxhat,\sshat)^{\trp}(\energy(\yyhat), \AA\yyhat - \bb, \cc(\yyhat))
%       \geq \energy(\yyhat) \geq \alpha^*
%   \end{align**}
% This sandwiching also tells us that $\yyhat$ and $(\xxhat,\sshat)$
% must be optimal for the primal and dual problems respectively.

% Clearly, we like to ensure that


% We'd like to show that
% for some dual-feasible $(\xxhat,\sshat)$ maximizing $L(\xx,
% \ss)$, then the hyperplane inequality doesn't have any slack, i.e. it
% holds with equality at some point, and furthermore, equality is
% obtained at some $(t, \vv, \xx) = (\energy(\yyhat), \AA \yyhat - \bb,
% \cc(\yyhat))$ where $\yyhat$ is feasible.





% When we proved weak duality in Section~\ref{sec:weakdual}, our reasoning

\begin{proof}[Proof of Theorem~\ref{thm:slaterstrongduality}.]
  For simplicity, our proof will assume that $S = \R^n$, but only a little
  extra work is required to handle the general case.

  Let's move to on finding two convex disjoints sets $A, B$ to enable the use of the separating hyperplane \autoref{th:sht}.

First set we define $A$, roughly speaking, as a multi-dimensional epigraph of
$G$. More precisely
\begin{equation*} A = \{(t, \vv, \uu) : \exists \yy \in S, t \geq \energy(\yy), \vv = \AA\yy - \bb, \uu \geq \cc(\yy)\}. \end{equation*}
Note that $A$ is a convex set. The proof is similar to the proof that the epigraph of a convex function is a convex set.
The optimal value of the primal program can be now written as
\begin{equation*} \alpha^* = \min_{(t,\veczero,\veczero) \in A} t. \end{equation*}
And we define another set $B$ of the same dimensionality as $A$ by
\begin{equation*} B := \{(r \in \R, \veczero \in \R^m, \veczero \in \R^k) : r < \alpha^* \}. \end{equation*}
This set $B$ is convex, as it is a ray.
An example of two such sets $A,B$ is illustrated in \autoref{fig:ab}.

We show that $A \cap B = \emptyset$ by contradiction.
Suppose $A, B$ are not disjoint; then there exists $\yy$ such that
\begin{equation*} (\energy(\yy), \AA\yy-\bb, \cc(\yy)) = (r, \veczero, \uu) \end{equation*}
with $\uu \leq \veczero$.
But this means that $\yy$ is feasible and  $\energy(\yy) = r < \alpha^*$; contradicting the optimality of $\alpha^*$.

% Now let's move back to proving that strong duality follows from Slater's condition.
%As Slater's condition requires the existence of a $\yy$ such that $\AA\yy = \bb$
To make things simpler,
we assume that our linear constraint matrix $\AA \in \R^{m\times n}$,
has full row rank and $m < n$ (but very little extra work is required to
deal with the remaining cases, which we omit).

As we just proved, $A$ and $B$ are convex and disjoint sets and hence the separating hyperplane theorem (\autoref{th:sht}) we introduced earlier in this chapter implies the existence a separating hyperplane.
This means there exists a normal
$\nn = (\tilde\rho, \xxtil, \sstil)$ and threshold
% \footnote{To prevent confusion, the threshold here is denoted $\mu$ instead of $alpha$} %
$\mu$ and with $A$ on one side, i.e.\
%\begin{equation*} (t, \vv, \uu) \in A \implies \begin{pmatrix}t \vv^{\trp} \uu^{\trp}\end{pmatrix}\begin{pmatrix}\tilde\rho \\ \xxtil \\ \sstil\end{pmatrix} \geq \alpha \end{equation*}
\begin{equation} \label{eq:A}
(t, \vv, \uu) \in A \implies (t, \vv, \uu)^{\trp}(\tilde\rho, \xxtil, \sstil) \geq \mu
\end{equation}
and the set $B$ on the other side:
\begin{equation} \label{eq:B}
(t, \vv, \uu) \in B \implies (t, \vv, \uu)^{\trp}(\tilde\rho, \xxtil, \sstil) \leq \mu.
\end{equation}

Now, we claim that $\sstil \geq 0$. Suppose $\sstil(i) < 0$, then for $\uu(i) \rightarrow \infty$ the threshold would grow unbounded, i.e.\ $\mu \rightarrow -\infty$ contradicting that the threshold $\mu$ is finite by the separating hyperplane theorem.
Similarly we claim $\tilde\rho \geq 0$, as if this were not the case, having $t \rightarrow \infty$ implies that $\mu \rightarrow -\infty$ again contradicting the finiteness of $\mu$.

From Equation~\eqref{eq:B} it follows that $t\tilde\rho \leq \mu$ for all $t < \alpha^*$ which implies that $t\tilde\rho \leq \mu$ for $t = \alpha^*$ by taking the limit. Hence we have $\alpha^*\tilde\rho \leq \mu$.
From $(t, \vv, \uu) \in A$ we get from  Equation~\eqref{eq:A}
\begin{equation*}
(\tilde\rho, \xxtil, \sstil)^{\trp}(t, \vv, \uu) \geq \mu \geq \alpha^*\tilde\rho
\end{equation*}
and thus
\begin{equation}\label{eq:rhotilde}
(\tilde\rho, \xxtil, \sstil)^{\trp}(\energy(\yy), \AA\yy-\bb,\cc(\yy)) \geq \alpha^*\tilde\rho.
\end{equation}


Now we consider two cases; starting with the ``good'' case where $\tilde\rho > 0$.
Dividing  Equation~\eqref{eq:rhotilde} by $\tilde\rho$ gives
\begin{equation*}
  \energy(\yy) + \frac{\xxtil^{\trp}}{\tilde\rho}(\AA\yy - \bb) + \frac{\sstil^{\trp}}{\tilde\rho}\cc(\yy) \geq \alpha^*.
\end{equation*}
Noting that the left hand side above is $L(\yy, \frac{\xxtil}{\tilde\rho}, \frac{\sstil}{\tilde\rho})$
and that the equation holds for arbitrary $\yy$; therefore also for the minimum we get
\begin{equation*} \min_{\yy}{L\left(\yy, \frac{\xxtil}{\tilde\rho}, \frac{\sstil}{\tilde\rho}\right)}\geq \alpha^* \end{equation*}
and hence via definition of $\beta^*$ finally
\begin{equation*} \beta^* \geq L\left(\frac{\xxtil}{\tilde\rho},\frac{\sstil}{\tilde\rho}\right) \geq \alpha^*. \end{equation*}

Next consider the ``bad'' case $\tilde\rho = 0$.
As $\alpha^*\tilde\rho \leq \mu$, we have $0 \leq \mu$.
From  Equation~\eqref{eq:A} we get
\begin{equation*}
  \cc(\yy)^{\trp}\ss + \xx^{\trp}(\bb-\AA\yy) \geq \mu \geq 0.
\end{equation*}
As Slater's condition holds, there is an interior point $\yytil$, i.e.\ it satisfies $\bb - \AA\yytil = \veczero$ and $\cc(\yytil) < \veczero$. Together with the equation above this yields
\begin{equation*} \cc(\yytil)^{\trp}\sstil + \xxtil^{\trp}\veczero \geq 0 \end{equation*}
which implies
$\cc(\yytil)^{\trp}\sstil \geq 0 $
and as $\cc(\yytil) < \veczero$ this means $\sstil = \veczero$.

As the normal $(\tilde\rho,\sstil,\xxtil)$ of the hyperplane can not be all zeroes, this means the last ``component'' $\xxtil$ must contain a non-zero entry, i.e.\ $\xxtil \neq \veczero$.
Furthermore $\xxtil^{\trp}(\bb - \AA\yytil) = \veczero$, $\cc(\yytil) < \veczero$ and $\AA$ has full row rank, hence there exists $\ddelta$ such that
\begin{equation*}
\xxtil^{\trp}(\bb-\AA(\yytil+\ddelta)) < \veczero \text{ and } \cc(\yytil + \ddelta) < 0.
\end{equation*}
This, however, means that there is a point in $A$ on the wrong side of the hyperplane, as
\begin{equation*}
(\tilde\rho, \xxtil, \sstil)^{\trp}(\energy(\yytil+\ddelta), \bb-\AA(\yytil+\ddelta), \cc(\yytil+\ddelta)) < 0
\end{equation*}
but the threshold is $\mu \geq 0$.
\end{proof}

\begin{remark*}
  Note that our reasoning about why $\ss \geq \veczero$ in the proof
  above is very similar to our reasoning for why the primal program
  can be written as Problem~\eqref{eq:primalminmax}.
\end{remark*}

\paragraph{Example.}
As an example of $A$ and $B$ as they appear in the above proof,
consider
\[
\min_{\substack{y \in (0,\infty) \\ 1/y
    -1 \leq 0}}
y^2
\]
 This leads to $\alpha^* = 1,
    y^* =1$, and $A = \setof{ (t,u): y \in (0,\infty) \text{ and } t >y^2
       \text{ and } u \geq 1/y-1}$, and $B = \setof{ (t,0) : t < 1}$ and the separating hyperplane normal is $\nn = (1,2)$.
These two sets $A,B$ are illustrated in \autoref{fig:ab}.
\begin{figure}[H]
  \centering
  \includegraphics[width=.9\textwidth]{fig/lec12-the-sets-a-b.jpg}
  \caption{Example of the convex sets $A$ and $B$ we wish to separate by hyperplane.}
\label{fig:ab}
\end{figure}



\subsection{KKT Revisited}
Let's come back to our earlier discussion of parallel gradients and the
KKT conditions.
 Consider a convex program \eqref{eq:primalprob} with an open set $S$
 as its domain.
Suppose $\yy^*$ is an optimizer of the primal problem and $\xx^*,
\ss^*$ for the dual, and suppose that strong duality holds.
We thus have
\[ L(\yy^*, \xx^*, \ss^*) = \alpha^* = \beta^*. \]
Because $L(\yy, \xx^*, \ss^*)$ is a convex function in $\yy$, it also
follows that as $\energy: S \to \R$ and $\cc$ are differentiable then we must have that the
gradient w.r.t. $\yy$ is zero, i.e.
\begin{equation}
  \left.\grad_{\yy}L(\yy, \xx^*,
    \ss^*)\right|_{\yy=\yy^*} = 0
  \label{eq:KKTgradAtOpt}
\end{equation}
This says exactly that the KKT gradient condition holds at $(\yy^*, \xx^*, \ss^*)$ .
% and plugging in
% \[ L(\yy, \xx^*, \ss^*) = \energy(\yy) + \xx^{\trp}(\bb - \AA\yy) + \ss^{\trp}\cc(\yy) \]
% yields
% \begin{equation*} \grad\energy(\yy) + \xx^{\trp}\grad_y(\bb-\AA\yy) + \ss^{\trp}\grad\cc(\yy) = \veczero. \end{equation*}
% And this connects to our point of the parallel gradients from the beginning of this section.

\paragraph{Complementary slackness.}
We can also see that
\begin{equation*}\energy(\yy^*) =  \alpha^* = \energy(\yy^*) + \xx^{\trp}(\bb - \AA\yy^*) + \ss^{\trp}\cc(\yy^*) = \energy(\yy^*) + \ss^{\trp}\cc(\yy^*) \end{equation*}
and hence when the $i$-th convex constraint is not active, i.e.\ $c_i(\yy^*) < 0$ the \emph{slack} must be zero, i.e.\ $\ss(i) = 0$.
Conversely if the slack is non-zero, that is $\ss(i) \neq 0$ implies
that the constraint is active, i.e.\ $c_i(\yy^*) = 0$.
This says precisely that the complementary slackness condition holds
at primal-dual optimal $(\yy^*, \xx^*, \ss^*)$.
Combined with our previous observation
Equation~\eqref{eq:KKTgradAtOpt}, we get the following result.

\begin{theorem}
  \label{thm:KKTgradAtOptStrongDual}
  Consider a convex program \eqref{eq:primalprob} with an open domain
  set $S$ and whose dual satisfies strong duality.  
Then  KKT conditions necessarily hold at primal-dual
  optimal $(\yy^*, \xx^*, \ss^*)$.
\end{theorem}

Theorem~\ref{thm:KKTgradAtOptStrongDual} combined with
Theorem~\ref{thm:slaterstrongduality} immediately
imply Proposition~\ref{pro:KKTnecessary}.

\paragraph{KKT is sufficient.}
We've seen that when strong duality holds, the KKT conditions are
necessary for optimality. In fact, they're also sufficient for optimality, as the
next theorem shows. And in this case, we do not need to assume
strong duality, as now it is implied by the KKT conditions. 
\begin{theorem}
   Consider a convex program \eqref{eq:primalprob} with an open domain
  set $S$.
   Then if the KKT conditions hold at $(\yytil,\xxtil,\sstil)$, they must be
   primal-dual optimal, and strong duality must hold.
\end{theorem}
\begin{proof}
  $\yytil$ is global minimizer of $\yy \mapsto L(\yy, \xxtil, \sstil)$, since this function is convex with vanishing gradient at $\yytil$. Hence,
  \[ L(\yytil, \xxtil, \sstil) = \inf_{\yy} L(\yy, \xxtil, \sstil) = L(\xxtil,\sstil) \leq \beta^*. \]
  On the other hand, due to primal feasibility and complementary slackness,
  \[ L(\yytil,\xxtil,\sstil) = \calE(\yytil) + \xxtil^\trp(\bb-\AA^\trp\yytil) + \sstil^\trp\cc(\yytil) = \calE(\yytil) \geq \alpha^*. \]
  Thus, $\beta^*\geq\alpha^*$. But also $\beta^*\leq\alpha^*$ by weak duality. Therefore, $\beta^*=\alpha^*$ and $\yytil,\xxtil,\sstil$ are primal/dual optimal.
\end{proof}


A good reference for basic convex duality theory is Boyd's free online book ``Convex
optimization'' (linked to on the course website).
It provides a number of different interpretations of duality. One
particularly interesting one comes from economics:
economists see the slack variables $\ss$ as prices for violating the constraints.

%\end{document}
%%% Local Variables:
%%% mode: latex
%%% TeX-master: "main"
%%% End:


