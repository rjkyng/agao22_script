% \documentclass[draft,11pt]{article}
%\documentclass[11pt]{article}
%%  RASMUS  packages
\usepackage{blindtext}
\usepackage[usenames,dvipsnames,svgnames,table]{xcolor}
\usepackage{enumitem}
\usepackage{caption}
\usepackage{placeins}
\usepackage{graphicx}

\usepackage{fancybox}
\usepackage{hyperref}

\usepackage{amsmath,amsfonts,amsthm,amssymb,xcolor}
% \let\amsboldsymbol\boldsymbol % if you want to see the bad
% boldsymbol for comparison
\usepackage{bm} % fixes boldsymbol

\usepackage{nicefrac}

\usepackage{float}


% switched to algorithm2e in lecture 9
% \usepackage{algorithm} %ctan.org\pkg\algorithms
% \usepackage{algorithmic}
% \usepackage{algorithmicx}
% \usepackage[noend]{algpseudocode}

\newtheorem{theorem}{Theorem}[section]
\newtheorem{corollary}[theorem]{Corollary}
\newtheorem{lemma}[theorem]{Lemma}
\newtheorem{observation}[theorem]{Observation}
\newtheorem{proposition}[theorem]{Proposition}
\newtheorem{claim}[theorem]{Claim}
\newtheorem{fact}[theorem]{Fact}
\newtheorem{assumption}[theorem]{Assumption}
\newtheorem{warning}[theorem]{Warning}
\newtheorem{conjecture}[theorem]{Conjecture}

\newtheorem{pseudotheorem}{Pseudotheorem}[section]
\newtheorem{pseudoclaim}[theorem]{Pseudoclaim}


\theoremstyle{definition}
\newtheorem{definition}[theorem]{Definition}
\newtheorem{remark}[theorem]{Remark}

\newtheorem*{theorem*}{Theorem}
\newtheorem*{corollary*}{Corollary}
\newtheorem*{conjecture*}{Conjecture}
\newtheorem*{lemma*}{Lemma}
\newtheorem*{thm*}{Theorem}
\newtheorem*{prop*}{Proposition}
\newtheorem*{obs*}{Observation}
\newtheorem*{definition*}{Definition}
\newtheorem*{example}{Example}
\newtheorem*{remark*}{Remark}
\newtheorem*{rec*}{Recommendation}

\newenvironment{fminipage}%
  {\begin{Sbox}\begin{minipage}}%
  {\end{minipage}\end{Sbox}\fbox{\TheSbox}}

\newenvironment{algbox}[0]{\vskip 0.2in
\noindent 
\begin{fminipage}{6.3in}
}{
\end{fminipage}
\vskip 0.2in
}


\let\muchl\ll

\def\pleq{\preccurlyeq}
\def\pgeq{\succcurlyeq}
\def\pge{\succ}
\def\ple{\prec}

\def\Approx#1{\approx_{#1}}

%\def\Span#1{\textbf{Span}\left(#1  \right)}
\def\bvec#1{{\mbox{\boldmath $#1$}}}


\newcommand\congestion{\mathit{cong}}

\def\prob#1#2{\mbox{Pr}_{#1}\left[ #2 \right]}
\def\pvec#1#2{\vec{\mbox{P}}^{#1}\left[ #2 \right]}
\def\expec#1#2{{\mathbb{E}}_{#1}\left[ #2 \right]}
\def\var#1{\mbox{\bf Var}\left[ #1 \right]}

\def\defeq{\stackrel{\mathrm{def}}{=}}
\def\setof#1{\left\{#1  \right\}}
\def\sizeof#1{\left|#1  \right|}


\def\trace#1{\mathrm{Tr} \left(#1 \right)}

\def\floor#1{\left\lfloor #1 \right\rfloor}
\def\ceil#1{\left\lceil #1 \right\rceil}

\def\dim#1{\mathrm{dim} (#1)}
\def\sgn#1{\mathrm{sgn} (#1)}

\def\union{\cup}
\def\intersect{\cap}
\def\Union{\bigcup}
\def\Intersect{\bigcap}

\def\abs#1{\left|#1  \right|}

\def\norm#1{\left\| #1 \right\|}
\def\smallnorm#1{\| #1 \|}

\newcommand\grad{\boldsymbol{\nabla}}
\newcommand\D[2]{D#1[#2]}

\newcommand*\diff{\mathop{}\!\mathrm{d}}
\newcommand*\Diff[1]{\mathop{}\!\mathrm{d^#1}}

\newcommand\ip[1]{\left< #1 \right>}


\newcommand{\sym}[1]{\mathrm{sym} (#1)}



\def\calC{\mathcal{C}}
\def\calD{\mathcal{D}}
\def\calE{\mathcal{E}}
\def\calF{\mathcal{F}}
\def\calG{\mathcal{G}}
\def\calL{\mathcal{L}}
\def\calS{\mathcal{S}}
\def\calT{\mathcal{T}}
\def\calM{\mathcal{M}}

\newcommand\calDD{\boldsymbol{\calD}}

\newcommand\DDelta{\boldsymbol{\mathit{\Delta}}}
\newcommand\Ppsi{\boldsymbol{\mathit{\Psi}}}
\newcommand\PPsi{\boldsymbol{\mathit{\Psi}}}
\newcommand\ppsi{\boldsymbol{\mathit{\psi}}}
\newcommand\pphi{\boldsymbol{\mathit{\phi}}}
\newcommand\PPhi{\boldsymbol{\Phi}}
%\newcommand\Llambda{\boldsymbol{\mathit{\Lambda}}}
\newcommand\LLambda{\boldsymbol{\mathit{\Lambda}}}
\newcommand\PPi{\boldsymbol{\Pi}}

\newcommand\ppi{\boldsymbol{\pi}}
\newcommand\cchi{\boldsymbol{\chi}}
\newcommand\aalpha{\boldsymbol{\alpha}}
\newcommand\bbeta{\boldsymbol{\beta}}
\newcommand\ggamma{\boldsymbol{\gamma}}
\newcommand\ddelta{\boldsymbol{\delta}}

\newcommand\rrho{\boldsymbol{\rho}}
\newcommand\xxi{\boldsymbol{\xi}}
%\newcommand\cchi{\boldsymbol{\chi}}

\newcommand\er{R_{\text{eff}}}


\def\aa{\pmb{\mathit{a}}}
\newcommand\bb{\boldsymbol{\mathit{b}}}
\newcommand\cc{\boldsymbol{\mathit{c}}}
\newcommand\dd{\boldsymbol{\mathit{d}}}
\newcommand\ee{\boldsymbol{\mathit{e}}}
\newcommand\ff{\boldsymbol{\mathit{f}}}
\renewcommand\gg{\boldsymbol{\mathit{g}}}
\newcommand\ii{\boldsymbol{\mathit{i}}}
\newcommand\jj{\boldsymbol{\mathit{j}}}
\newcommand\kk{\boldsymbol{\mathit{k}}}
\renewcommand\ll{\boldsymbol{\mathit{l}}}
\newcommand\pp{\boldsymbol{\mathit{p}}}
\newcommand\qq{\boldsymbol{\mathit{q}}}
\newcommand\bs{\boldsymbol{\mathit{s}}}
\newcommand\nn{\boldsymbol{\mathit{n}}}
\newcommand\rr{\boldsymbol{\mathit{r}}}
\renewcommand\ss{\boldsymbol{\mathit{s}}}
\def\tt{\boldsymbol{\mathit{t}}}
\newcommand\uu{\boldsymbol{\mathit{u}}}
\newcommand\vv{\boldsymbol{\mathit{v}}}
\newcommand\ww{\boldsymbol{\mathit{w}}}
\newcommand\yy{\boldsymbol{\mathit{y}}}
\newcommand\zz{\boldsymbol{\mathit{z}}}
\newcommand\xx{\boldsymbol{\mathit{x}}}

\newcommand\veczero{\boldsymbol{0}}
\newcommand\vecone{\boldsymbol{1}}

\newcommand\matzero{\boldsymbol{0}}
\newcommand\matone{\boldsymbol{1}}

\newcommand{\matlow}{\boldsymbol{\mathit{{\mathcal{L}}}}}
\newcommand{\matlowtil}{\boldsymbol{\mathit{\widetilde{\mathcal{L}}}}}
\newcommand{\matlowhat}{\boldsymbol{\mathit{\widehat{\mathcal{L}}}}}

\newcommand{\matup}{\boldsymbol{\mathit{{\mathcal{U}}}}}


\renewcommand\AA{\boldsymbol{\mathit{A}}}
\newcommand\BB{\boldsymbol{\mathit{B}}}
\newcommand\CC{\boldsymbol{\mathit{C}}}
\newcommand\DD{\boldsymbol{\mathit{D}}}
\newcommand\EE{\boldsymbol{\mathit{E}}}
\newcommand\GG{\boldsymbol{\mathit{G}}}
\newcommand\HH{\boldsymbol{{H}}}
\newcommand\II{\boldsymbol{\mathit{I}}}
\newcommand\JJ{\boldsymbol{\mathit{J}}}
\newcommand\KK{\boldsymbol{\mathit{K}}}
\newcommand\NN{\boldsymbol{\mathit{N}}}
\newcommand\MM{\boldsymbol{\mathit{M}}}
\newcommand\LL{\boldsymbol{\mathit{L}}}
\newcommand\PP{\boldsymbol{\mathit{P}}}
\newcommand\RR{\boldsymbol{\mathit{R}}}
\renewcommand\SS{\boldsymbol{\mathit{S}}}
\newcommand\TT{\boldsymbol{\mathit{T}}}
\newcommand\UU{\boldsymbol{\mathit{U}}}
\newcommand\WW{\boldsymbol{\mathit{W}}}
\newcommand\VV{\boldsymbol{\mathit{V}}}
\newcommand\XX{\boldsymbol{\mathit{X}}}
\newcommand\YY{\boldsymbol{\mathit{Y}}}



\newcommand\MMtil{\boldsymbol{\mathit{\tilde{M}}}}
\newcommand\AAtil{\boldsymbol{\mathit{\tilde{A}}}}
\newcommand\BBtil{\boldsymbol{\mathit{\tilde{B}}}}
\newcommand\LLtil{\boldsymbol{\mathit{\tilde{L}}}}
\newcommand\MMtilde{\boldsymbol{\mathit{\tilde{M}}}}
\newcommand\XXtil{\boldsymbol{\mathit{\tilde{X}}}}

\newcommand\AAn{\boldsymbol{\mathcal{A}}}
\newcommand\ZZ{\boldsymbol{\mathit{Z}}}

\newcommand\AAhat{\boldsymbol{\widehat{\mathit{A}}}}
\newcommand\AAapprox{\boldsymbol{\widetilde{\mathit{A}}}}
\newcommand\DDhat{\boldsymbol{\widehat{\mathit{D}}}}
\newcommand\DDapprox{\boldsymbol{\widetilde{\mathit{D}}}}
\newcommand\LLhat{\boldsymbol{\widehat{\mathit{L}}}}
\newcommand\LLapprox{\boldsymbol{\widetilde{\mathit{L}}}}
\newcommand\MMhat{\boldsymbol{\widehat{\mathit{M}}}}
\newcommand\MMapprox{\boldsymbol{\widetilde{\mathit{M}}}}
\newcommand\ZZhat{\boldsymbol{\widehat{\mathit{Z}}}}

\newcommand\DDtil{\boldsymbol{\widetilde{\mathit{D}}}}

\newcommand\fftil{\boldsymbol{\tilde{\mathit{f}}}}
\newcommand\sstil{\boldsymbol{\tilde{\mathit{s}}}}
\newcommand\xxtil{\boldsymbol{\tilde{\mathit{x}}}}
\newcommand\yytil{\boldsymbol{\tilde{\mathit{y}}}}
\newcommand\wwtil{\boldsymbol{\tilde{\mathit{w}}}}

\newcommand\ddeltatil{\boldsymbol{\tilde{\mathit{\delta}}}}


\newcommand\Otil{\widetilde{O}}

\newcommand\xhat{{\hat{{x}}}}
\newcommand\uhat{{\hat{{u}}}}
\newcommand\uuhat{\boldsymbol{\mathit{\hat{u}}}}
\newcommand\vhat{{\hat{{v}}}}
\newcommand\what{{\hat{{w}}}}

\newcommand\Ghat{{\widehat{{G}}}}
\newcommand\GGhat{\boldsymbol{\widehat{G}}}

\newcommand\Ehat{{\widehat{{E}}}}


\newcommand\R{\mathbb{R}}
\newcommand\N{\mathbb{N}}

\newcommand\ffhat{\boldsymbol{\hat{\mathit{f}}}}

\newcommand\cchat{\boldsymbol{\widehat{\mathit{c}}}}
\newcommand\sshat{\boldsymbol{\mathit{\widehat{s}}}}
\newcommand\xxhat{\boldsymbol{\mathit{\widehat{x}}}}
\newcommand\yyhat{\boldsymbol{\widehat{\mathit{y}}}}
\newcommand\xxbar{\overline{\boldsymbol{\mathit{x}}}}
\newcommand\yybar{\overline{\boldsymbol{\mathit{y}}}}
\newcommand\xxstar{{\boldsymbol{\mathit{x}}^{*}}}
\newcommand\yystar{{\boldsymbol{\mathit{y}}^{*}}}


\newcommand\ffbar{\overline{\boldsymbol{\mathit{f}}}}


\newcommand\ddeltabar{\boldsymbol{\bar{\mathit{\delta}}}}
\newcommand\ddeltahat{\boldsymbol{\hat{\mathit{\delta}}}}



\newcommand\energy{\mathcal{E}}


\newcommand{\todo}[1]{{\bf \color{red} TODO: #1}}
\newcommand{\todolow}[1]{{\bf \color{orange} TODOLOW: #1}}

\newcommand{\richard}[1]{{\bf \color{green} Richard: #1}}
\newcommand{\rasmus}[1]{{\bf \color{olive} Rasmus: #1}}
\newcommand{\ahad}[1]{{\bf \color{olive} Ahad: #1}}


\newcommand{\expct}[2]{{}\mathop{\mathbb{E}}_{#1}\left[#2\right]}
\newcommand{\E}[1]{\mathop{{}\mathbb{E}}\left[#1\right]}
%% https://tex.stackexchange.com/questions/56765/getting-the-expectation-symbol-to-behave-like-sum-instead-of-sigma
%% extra {} inside mathop gives nicer (standard) vertical placement of E
\newcommand{\Var}[1]{\mathop{{}Var}\left[#1\right]}
\newcommand{\Ex}[1]{{}\mathop{\mathbb{E}}_{#1}}
\newcommand\tr{\mathrm{Tr}}


%%%%% LINEAR ALGEBRA

\newcommand{\schurto}[2]{\ensuremath{\textsc{Sc}\!\left[#1\right]_{#2}}}
% \newcommand{\schurto}[2]{\ensuremath{\textsc{Sc}\left[#1\right]_{#2}}}

%{$\textsc{Sc}\left[#1\right]_{#2}$}
% \newcommand{\schurto}[2]{\textsc{Sc}\left(#1, #2\right)}
%\newcommand{\schurto}[2]{\left[{#1}\right]_{\to #2}}
\renewcommand{\sc}[2]{\schurto{#1}{#2}}

%transpose
\newcommand{\trp}{\top}

%pseudoinverse
\newcommand{\pinv}{+}


\newcommand{\proj}{\PPi}

\DeclareMathOperator{\nnz}{nnz}

\DeclareMathOperator*{\argmin}{arg\,min}
\DeclareMathOperator*{\argmax}{arg\,max}
\DeclareMathOperator*{\val}{val}

\DeclareMathOperator*{\diag}{diag}
\DeclareMathOperator*{\Span}{span}

%\DeclareMathOperator*{\ker}{ker}
\DeclareMathOperator*{\im}{im}

%%%%% LAYOUT

\newenvironment{tight_enumerate}{
\begin{enumerate}
 \setlength{\itemsep}{2pt}
 \setlength{\parskip}{1pt}
}{\end{enumerate}}
\newenvironment{tight_itemize}{
\begin{itemize}
 \setlength{\itemsep}{2pt}
 \setlength{\parskip}{1pt}
}{\end{itemize}}
\newenvironment{tight_description}{
\begin{description}
 \setlength{\itemsep}{2pt}
 \setlength{\parskip}{1pt}
}{\end{description}}

\newcommand*{\vertbar}{\rule[-1ex]{0.5pt}{2.5ex}}
\newcommand*{\horzbar}{\rule[.5ex]{2.5ex}{0.5pt}}

%Basics
\newcommand{\new}[1]{{\em #1\/}}		% New term (set in italics).

\newcommand{\boxwidth}{\dimexpr\linewidth-2em\relax}

\newcommand{\boxdef}[1]
{
\fbox{
\begin{minipage}{\boxwidth}
\begin{definition}
{#1}
\end{definition}
\end{minipage}
}
}

\newcommand{\boxthm}[1]
{
\fbox{
\begin{minipage}{\boxwidth}
\begin{theorem}
{#1}
\end{theorem}
\end{minipage}
}
}

\newcommand{\boxfact}[1]
{
\fbox{
\begin{minipage}{\boxwidth}
%\begin{theorem*}
\emph{{#1}
%\end{theorem*}
}
\end{minipage}
}
}

%\newcommand{\handout}[5]{
  \noindent
  \begin{center}
  \framebox{
    \vbox{
            \hbox to 6.30in { {\bf Advanced Graph Algorithms and Optimization} \hfill #2 }
            \vspace{5mm}
            \hbox to 6.30in { {\Large \hfill #5  \hfill} }
            \vspace{3mm}
            \hbox to 6.30in { {\em #3 \hfill #4} }
    }
  }
  \end{center}
  \vspace*{4mm}
}

\newcommand{\lecture}[4]{\handout{#1}{#2}{#3}{Lecture #1}{#4}}
\newcommand{\homework}[3]{\handout{#1}{#2}{#3}{Problem Set #1}}
\newcommand{\gradedhomework}[3]{\handout{#1}{#2}{#3}{Graded Homework #1}}
\newcommand{\sect}[3]{\handout{#1}{#2}{#3}{Section #1}}




% 1-inch margins, from fullpage.sty by H.Partl, Version 2, Dec. 15, 1988.
\topmargin 0pt
\advance \topmargin by -\headheight
\advance \topmargin by -\headsep
\textheight 8.9in
\oddsidemargin 0pt
\evensidemargin \oddsidemargin
\marginparwidth 0.5in
\textwidth 6.5in

\parindent 0in
\parskip 1.5ex
 
%\usepackage{amsmath}
%\usepackage{amssymb}
%\usepackage{amsthm}
%\usepackage{graphicx}
%\usepackage{float}
%\usepackage[ruled,vlined]{algorithm2e}
%\SetKwBlock{Repeat}{repeat}{}
%%% for this lecture
%\newcommand{\gap}{\text{gap}}
% \newtheorem{theorem}{Definition}
%\interfootnotelinepenalty=10000
% \newtheorem{lemma}{Lemma}

%\usepackage{tikz}
%\usetikzlibrary{arrows}

\chapter{Classical
  Algorithms for Maximum Flow II}
\label{cha:maxflow2}

%\begin{document}

\sloppy
%\lecture{11 --- Wednesday, May 6th}
%{Spring 2020}{Rasmus Kyng, Scribe: Timon Knigge}{Classical
  %Algorithms for Maximum Flow II}

\section{Overview}
In this chapter we continue looking at classical max flow algorithms. We derive Dinic's algorithm which
(unlike Ford-Fulkerson) converges in a polynomial number of iterations. We will also show faster
convergence on unit capacity graphs.
The setting is the same as last chapter: we have a directed graph $G = (V, E, \cc)$ with
positive capacities
on the edges. Our goal is to solve the maximum flow problem on this graph for a given source $s$ and sink
$t \neq s$:

\begin{align}
  \begin{aligned}
    \max_{F>0} \;F\qquad\text{s.t.}\quad&\BB\ff = F\bb_{s,t} \\
    &\veczero\leq \ff\leq \cc
  \end{aligned}
\end{align}

\section{Blocking Flows}
Let $\ff$ be any feasible flow in $G$ and let $G_{\ff}$ be its residual graph.
Observe that we can partition the
vertices of $G_{\ff}$ into `layers': first $s$ itself, then all vertices reachable from $s$ in one hop, then all
vertices reachable in two hops, etc. For each vertex $v\in V$ define its \textbf{level} in $G_{\ff}$ as the
length of the shortest path in $G_{\ff}$ from $s$ to $v$, denoted by $\ell_{G_{\ff}}(v)$ (or just $\ell(v)$ if the
graph is clear from context). An edge $(u, v) \in E$ can only take you up `one level': if
$\ell(v) \geq 2+\ell(u)$ this would imply we can find a shorter $s$-$v$ path by appending $(u, v)$ to the
shortest $s$-$u$ path. However, edges can `drop down' multiple levels (or be contained in the same level).

A key strategy in Dinic's algorithm will be to focus on `short' augmenting paths. We can use the levels we
defined above to isolate a subgraph of $G_{\ff}$ containing all information we need to find shortest paths.
\begin{definition}
  Call an edge $(u, v)$ \textbf{admissible} if $\ell(u)+1 = \ell(v)$. Let $L$ be the set of
  admissible edges in $G_{\ff}$ and define the \textbf{level graph} of $G_{\ff}$ to be the subgraph
  induced by these edges.
\end{definition}

\begin{definition}
  Define a \textbf{blocking flow} in a residual graph $G_{\ff}$ to be a flow $\hat{\ff}$ feasible in $G_{\ff}$,
  such that $\hat{\ff}$ only uses admissible edges. Furthermore we require that for any $s$-$t$ path in
  the level graph of $G_{\ff}$, $\hat{\ff}$ saturates at least one edge on this path.
\end{definition}

The last condition makes a blocking flow `blocking': it blocks any shortest augmenting path in $G_{\ff}$ by
saturating one of its edges. Note however that a blocking flow is not necessarily a maximum flow in the
level graph:

\begin{figure}
  \centering
  \begin{tikzpicture}
    \tikzset{vertex/.style = {shape=circle,draw,minimum size=1.5em}}
    \tikzset{edge/.style = {->,> = latex'}}

    \node[vertex] (a) at (0, 0) {$s$};
    \node[vertex] (b) at (2, -1) {};
    \node[vertex] (c) at (2, 1) {};
    \node[vertex] (d) at (4, 0) {};
    \node[vertex] (e) at (4, 2) {};
    \node[vertex] (f) at (6, 1) {$t$};

    \draw[edge] (a) -- (b) node [midway, below] {$0/1$};
    \draw[edge] (b) -- (d) node [midway, below] {$0/1$};
    \draw[edge] (c) -- (e) node [midway, above] {$0/1$};
    \draw[edge] (e) -- (f) node [midway, above] {$0/1$};

    \draw[edge, line width=2pt] (a) -- (c) node [midway, above] {$1/1$};
    \draw[edge, line width=2pt] (c) -- (d) node [midway, above] {$1/1$};
    \draw[edge, line width=2pt] (d) -- (f) node [midway, below] {$1/1$};

  \end{tikzpicture}
  \caption{A blocking flow is not a maximum flow.}
\end{figure}

\section{Dinic's Algorithm}
We can now formulate Dinic\footnote{Sometimes also transliterated as Dinitz.}'s algorithm: start with
$\ff = \veczero$, and then repeatedly add to $\ff$ a blocking flow $\hat{\ff}$ in $G_{\ff}$, until no more $s$-$t$ paths
exist in $G_{\ff}$.

Note that by doing a path decomposition on $\hat{\ff}$ and adding these paths to $\ff$ one by one, we see that
our algorithm is a `special case' of Ford-Fulkerson (with a particular augmenting path strategy) and hence
inherits all the behaviors/bounds we proved in the last chapter.

\begin{lemma}
  Let $\ff$ be a feasible flow, $\hat{\ff}$ a blocking flow in $G_{\ff}$ and define $\ff' = \ff+\hat{\ff}$ (think
  of $\ff$ and $\ff'$ to be the flows at some steps $k$ and $k+1$ in the algorithm). Then
  $\ell_{G_{\ff'}}(t) \geq \ell_{G_{\ff}}(t) + 1$.
\end{lemma}
\begin{proof}
  Let $L, L'$ be the edge sets of the level graphs of $G_{\ff}, G_{\ff'}$
  respectively.
  % \todo{I didn't really
  %   get the proof in the class for this part very well so I wrote my own. Feel
  %   free to change, I hope it's okay though.}
  We would now like to show that $d_{L'}(s, t) \geq 1 + d_L(s, t)$. Let's first assume that in fact
  $d_{L'}(s, t) < d_L(s, t)$. Now take a shortest $s$-$t$ path in the level graph of
  $G_{\ff'}$, say $s=v_0, v_1, \dots, v_{d_{L'}(s, t)} =  t$.
  Let $v_j$ be the first vertex along the path such that
  $d_{L'}(s, v_j) < d_L(s, v_j)$.
  As $d_{L'}(s, t) < d_L(s, t)$, such a $v_j$ must exist.

  We'd like to understand the edges in the level
  graph $L'$.
  Let $E(G_{\ff})$ denote the edges of the $G_{\ff}$,
  and let $\operatorname{rev}(L)$ denote the set of
  reversed edges of $L$.
  The level graph edges of $G_{\ff'}$ must satisfy
  $L' \subseteq L \union
  \operatorname{rev}(L) \union (E(G_{\ff}) \setminus
  L)$.

  In our $s$-$t$ path in $L'$, the vertex $v_j$ is reached by an edge from $v_{j-1}$,
  and the level of $v_{j-1}$ in $G_{\ff}$ and $G_{\ff'}$
  are the same, so the edge $(v_{j-1},v_j) \in L$ skips
  at least one level forward  in $G_{\ff}$.
  But, edges in $L$ do not skip a level in
  $G_{\ff}$, and edges in
  $\operatorname{rev}(L)$ or
  $E(G_{\ff}) \setminus L$ do not move from a lower level to higher
  level in $G_{\ff}$.
  So this edge cannot exist and we have reached a contradiction.
  % Write down the levels of the vertices in
  % $G_{\ff}$ along the path,
  % then
  % there are more levels in $G_{\ff}$ than
  % there are vertices in the path, so there must a be
  % level in $G$
  % so by the pigeonhole principle there must be a level in $G_{\ff}$ that
  % the path jumps over, i.e. there is some edge that jumps up more than one
  % level in $G_{\ff}$.
  % However, we claim this is impossible. Such an edge cannot exist in $G_{\ff}$ by definition of the level
  % graph. The only other edges in $G_{\ff'}$ are reversals of edges used in the blocking flow $\hat{\ff}$,
  % but by construction this flow only uses admissible edges in $G_{\ff}$ (i.e. edges that `step up' one
  % level). Hence the edge $(v_i, v_{i+1})$ cannot exist and we have arrived at a contradiction.

  Now suppose that $d_{L'}(s, t) = d_L(s, t)$. This means there is an $s$-$t$ path in $L'$ using
  edges in $L$ -- but such a path must contain an edge saturated by
  the blocking flow $\hat{\ff}$.

  (Note: the reason this path must use only edges in $L$ is similar to
  the previous case: the other possible types of edges do not move
  to higher levels in $G_{\ff}$ at all, making the path too long if we use any of them.)

  So we must have $d_{L'}(s, t) \geq 1 + d_L(s, t)$.
\end{proof}
An immediate corollary of this lemma is the convergence of Dinic's algorithm:
\begin{theorem}
  Dinic's algorithm terminates in $O(n)$ iterations.
\end{theorem}
\begin{proof}
  By the previous lemma, the level of $t$ increases by at least one each iteration. A shortest path
  can only contain each vertex once (otherwise it would contain a cycle) so the level of any vertex
  is never more than $n$.
\end{proof}

For graphs with unit capacities ($c = 1$) we can prove even better bounds on the number of iterations.
\begin{theorem}
  On unit capacity graphs, Dinic's algorithm terminates in
  $$O\left(\min\{m^{1/2}, n^{2/3}\}\right)$$ iterations.
\end{theorem}
\begin{proof}
  We prove the two bounds separately.
  \begin{enumerate}
  \item Suppose we run Dinic's algorithm for $k$ iterations, obtaining a flow $\ff$ that is
    feasible but not necessarily yet optimal. Let $\tilde{\ff}$ be an optimal flow in
    $G_{\ff}$ (i.e. $\ff+\tilde{\ff}$ is a maximum flow for the whole graph),
    and consider a path decomposition of $\tilde{\ff}$. Since after $k$ iterations
    any $s$-$t$ path has length $k$ or more, we use up a total capacity of at least
    $\text{val}(\tilde{\ff})k$ across all edges. But the edges in $G_{\ff}$ are either edges
    from the original graph $G$ or their reversals (but never both) meaning the total
    capacity of $G_{\ff}$ is at most $m$, hence $\text{val}(\tilde{\ff}) \leq m/k$.

    Recalling our earlier observation that our algorithm is a special case of
    Ford-Fulkerson, this implies that our algorithm will terminate after at most
    another $m/k$ iterations. Hence the number of iterations is bounded by $k+m/k$ for
    any $k>0$. Substituting $k=\sqrt{m}$ gives the first desired bound.
  \item Suppose again that we run Dinic's algorithm for $k$ iterations obtaining a flow $\ff$.
    The level graph of $G_{\ff}$ partitions the vertices into sets
    $D_i = \{u\mid\ell(u) = i\}$ for $i\geq 0$.
    As shown before, the sink $t$ must be at a level of at
    least $k$, meaning we have at least this many non-empty levels starting from
    $D_0 = \{s\}$. To simplify, discard all vertices and levels beyond $D_{\ell(t)}$.

    Now, consider choosing a level $I$ uniformly at random from
    $\setof{1,\ldots,k}$.
    Since there are at most $n-1$ vertices in total across these levels,
    $\E{\abs{D_I}} \leq (n-1)/k$, and by Markov's inequality
    \[
      \Pr[ \abs{D_I} \geq 2n/k ] \leq
      \frac{(n-1)/k}{2n/k} < \frac{1}{2}
      .
    \]
    % \todo{I found this a bit easier to follow than Markov, feel free to change (again).}
    % divide the levels into two groups: those with $|D_i|\leq 2n/k$ and those with
    % $|D_i| > 2n/k$. If the latter group contained at least $k/2$ levels then together
    % they would contain strictly more than $(2n/k)\cdot(k/2) = n$ vertices, which is a
    % contradiction (note the $D_i$ are disjoint), so more than $k-k/2 = k/2$ levels
    % satisfy the first condition,

    Thus strictly more than $k/2$ of the levels $i \in \setof{1,\ldots,k}$
    have $\abs{D_i} < 2n/k$ and so there must be two adjacent levels $j$, $j+1$
    for which this upper bound on the size holds.
    There can be at most
    $|D_j|\cdot|D_{j+1}| \leq 4n^2/k^2$ edges between these levels, and by the min-cost
    max-flow theorem we saw in the previous chapter, this is an upper bound on the flow
    in $G_{\ff}$, and hence on the number of iterations still needed for our algorithm to
    terminate.

    This means the number of iterations is bounded by $k + 4n^2/k^2$ for any $k>0$,
    which is $O(n^{2/3})$ at $k=2n^{2/3}$.
  \end{enumerate}
\end{proof}

\section{Finding Blocking Flows}
 \label{sec:findBlockingFlows} 
What has been missing from our discussion so far is the process of actually finding a blocking flow.
We can achieve this using repeated depth-first search. We repeatedly do a search in the level graph (so
only using edges $L$) for $s$-$t$ paths and augment these. We erase edges whose subtrees have been
exhausted and do not contain any augmenting paths to $t$. Pseudocode for the algorithm is given below.

\begin{algorithm}[H]
  \SetAlgoLined
  $\ff \leftarrow \veczero$\;
  $H \leftarrow L$\;
  \Repeat{
    $P\leftarrow\text{\textsc{Dfs}}(s, H, t)$\;
    \eIf{$P\neq\emptyset$}{
      Let $\hat{\ff}$ be a flow that saturates path $P$.\\
      $\ff\leftarrow \ff+\hat{\ff}$\;
      Remove from $H$ all edges saturated by $\hat{\ff}$.
    }{
      \Return{\ff}\;
    }
  }
  \caption{\textsc{FindBlockingFlow}}
\end{algorithm}

\begin{algorithm}[H]
  \SetAlgoLined
  \If{$u=t$}{\Return{\textnormal{the path $P$ on the dfs-stack.}}}
  \For{$(u,v)\in H$}{
    $P\leftarrow\text{\textsc{Dfs}}(v, H, t)$\;
    \eIf{$P\neq\emptyset$}{
      \Return{$P$}\;
    }{
      Erase $(u, v)$ from $H$.
    }
  }
  \Return{$\emptyset$}\;
  \caption{\textsc{Dfs}(u, H, t)}
\end{algorithm}

\begin{lemma}
  For general graphs, \textsc{FindBlockingFlow} returns a blocking flow in $O(nm)$ time.
  Hence Dinic's algorithm runs in $O(n^2 m)$ time on general capacity graphs.
\end{lemma}
\begin{proof}
  % We sketch an approach to amortized analysis of the running time.
  First, consider the amount of work spent pushing edges onto the
  stack which eventually results in augmentation along an $s$-$t$ path
  consisting of those edges (i.e.\ adding flow along that path).
  Since each augmenting path saturates at least one edge,  we do at most $m$ augmentations. Each
  path has length at most $n$.
  Thus the total amount of work pushing these edges to the stack, and removing them
  from the stack upon augmentation, and deleting saturated edges, can
  be bounded by $O(mn)$.
  Now consider the work spent pushing edges onto the stack which are
  later deleted because we ``retreat'' after not finding an $s$-$t$.
  An edge can only be pushed onto the stack once in this way, since it
  is then deleted. So the total amount spent pushing edges to the
  stack and deleting them this way is $O(m)$.
   % First, consider the amount of work spent pushing edges onto the
  % Since each augmenting path saturates at least one edge, there are at most $m$ augmenting paths. Each
  % path has length at most $n$.
  % Hence the calls to \textsc{Dfs} that result in augmenting paths account for
  % $O(mn)$ work jointly. The calls to \textsc{Dfs} that do not result in an augmenting path together account
  % for $O(m)$ work because each such call erases an edge, meaning across the whole algorithm run there are at
  % most $m$ such calls.
\end{proof}
\begin{lemma}
  For unit capacity graphs, \textsc{FindBlockingFlow} returns a blocking flow in $O(m)$ time. Hence Dinic's
  algorithm runs in $O\left(\min\{m^{3/2}, mn^{2/3}\}\right)$ time on unit capacity graphs.
\end{lemma}
\begin{proof}
  When our depth-first search traverses some edge $(u, v)$, one of two things will happen: either we find no
  augmenting path in the subtree of $v$, leading to the erasure of this edge, or we find an augmenting path
  which will necessarily saturate $(u, v)$, again leading to its erasure. This means each edge will be
  traversed at most once by the depth-first search.
\end{proof}

Another interesting bound on the runtime, which we will not prove here, is that Dinic's algorithm will run in
$O\left(\abs{E}\sqrt{\abs{V}}\right)$ time on bipartite matching graphs.

\section{Minimum Cut as a Linear Program}
Finally we show that minimum cut may be formulated as a linear program. For a subset $S \subseteq V$ write
$c_G(S)$ for the sum of $\cc(e)$ over all edges $e = (u, v)$ that cross the cut, i.e.\ $u \in S, v\not\in S$.
Note that `reverse' edges are not counted in this cut. The minimum cut problem asks us to find some $S\subseteq V$
such that $s\in S$ and $t\not\in S$ such that $c_G(S)$ is minimal.
\begin{align}
  \begin{aligned}
    \min_{S\subseteq V} \quad &c_G(S) \\
    s.t.\quad & s \in S \\
    &t\not\in S\\
  \end{aligned}
  \label{prog0}
\end{align}

We claim this is equivalent to the following
minimization problem:
\begin{align}
  \begin{aligned}
    \min_{\xx \in \R^{V}} \quad &\sum_{e\in E} \cc(e) \max\left\{\bb_e^\trp \xx, 0\right\} \\
    s.t. \quad & \xx(s) = 0 \\
    & \xx(t) = 1 \\
    &\veczero\leq \xx\leq \vecone \\
  \end{aligned}
  \label{prog1}
\end{align}
Recall $\bb_e^\trp \xx = \xx(v) - \xx(u)$ for $e = (u, v)$.
We can rewrite this as a proper linear program by
introducing extra variables for the maximum:
\begin{align}
  \begin{aligned}
    \min_{\xx \in \R^{V}, \uu \in \R^{E}} \quad & \cc^\trp \uu \\
    s.t. \quad & \bb_{s,t}^\trp\xx = 1 \\
    &\veczero\leq \xx\leq \vecone \\
    &\uu \geq \veczero \\
    &\uu \geq \BB^\trp \xx \\
  \end{aligned}
  \label{prog2}
\end{align}
And, in fact, we'll also see that we don't need the constraint
$\veczero\leq \xx\leq \vecone$, leading to the following simpler
program:
\begin{align}
  \begin{aligned}
    \min_{\xx \in \R^{V}, \uu \in \R^{E}} \quad & \cc^\trp \uu \\
    s.t. \quad & \bb_{s,t}^\trp\xx = 1 \\
    &\uu \geq \veczero \\
    &\uu \geq \BB^\trp \xx \\
  \end{aligned}
  \label{prog3}
\end{align}

\begin{lemma}
  Programs \eqref{prog0}, \eqref{prog2}, and \eqref{prog3} have equal optimal values.
\end{lemma}
\begin{proof}
  We start by considering equality between optimal values of Program
  \eqref{prog0} and Program \eqref{prog2}
  Let $S$ be an optimal solution to Program~\eqref{prog0} and take $\xx = \vecone_{V\setminus S}$. Then $\xx$ is a
  feasible solution to Program~\eqref{prog2} with cost equal to $c_G(S)$.

  Conversely, suppose $\xx$ is an optimal solution to Program~\eqref{prog2}. Let $\alpha$ be uniform in $[0,1]$ and define
  $S = \{v\in V\mid\xx(v) \leq \alpha\}$. This is called a threshold cut.
  We can verify that
  \[ \Pr\left[\text{$e$ is cut by $S$}\right] = \max\{\bb_e^\trp \xx, 0\}. \]
  and hence $\mathbb{E}_t [c_G(S)]$ is exactly the optimization function of \ref{prog1} (and hence
  \ref{prog2}).
  Since at least one outcome must do as well as the average, there is a subset $S\subseteq V$ achieving this value
  (or less).

  We can use the same threshold cut to show that we can round a
  solution to Program~\eqref{prog3} to an equal or smaller value cut
  feasible for Program~\eqref{prog0}. The only difference is that in
  this case, we get
  \[
    \Pr\left[\text{$e$ is cut by $S$}\right] \leq \max\{\bb_e^\trp \xx, 0\}
  \]
  which is still sufficient.
\end{proof}


%%% Local Variables:
%%% mode: latex
%%% TeX-master: "agao21_script"
%%% End:

%\end{document}
